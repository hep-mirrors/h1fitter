\makeatletter
\def\comsp{\@ifnextchar,\relax{\@ifnextchar\ \relax{\@ifnextchar:\relax{\@ifnextchar.\relax\ }}}}
\makeatother
\DeclareRobustCommand{\ie}{{\it i.e.}\comsp}
\DeclareRobustCommand{\eg}{{\it e.g.}\comsp}
\DeclareRobustCommand{\cf}{{\it cf.}\comsp}
\DeclareRobustCommand{\etal}{{\it et al.}\comsp}
\DeclareRobustCommand\bs{\ensuremath{\backslash}}
\newcommand\ssp{\ifmmode\relax\else\comsp\fi}

%\newcommand\Eq[1]{Eq.~(\ref{#1})}
\newcommand\Eq[1]{(\ref{#1})}
\newcommand\Fig[1]{Fig.~\ref{#1}}
\DeclareRobustCommand\NF{\ensuremath{N_{\rm f}}\ssp}
% \DeclareRobustCommand\NC{\ensuremath{N_{\rm c}}\ssp}
\DeclareRobustCommand\GeV{\ensuremath{{\rm GeV}}\ssp}
% \DeclareRobustCommand\Vstat{\ensuremath{V^{\rm (stat)}}\ssp}
% \DeclareRobustCommand\Vsys{\ensuremath{V^{\rm (sys)}}\ssp}
\DeclareRobustCommand\FL{\ensuremath{F_{\mathrm{L}}}\ssp}
\DeclareRobustCommand\FT{\ensuremath{F_{\mathrm{T}}}\ssp}
\newcommand\AP {{\cal P}}

\def \beq{\begin{equation}}
\def \eeq{\end{equation}}
\def \beqa{\begin{eqnarray}}
\def \eeqa{\end{eqnarray}}
\def \beqal{\begin{subequations}\begin{eqnarray}}
\def \eeqal{\end{eqnarray}\end{subequations}}

\let\optspace=\ssp
% \newcommand{\xh}{\hat x}
\DeclareRobustCommand{\as}[1]{\ensuremath{\alpha_{\rm s}(#1^2)}\optspace}
\newcommand{\asotp}{\ensuremath{\frac{\alpha_{\rm s}}{2\pi}}\optspace}
\newcommand{\Sgl}[1]{\ensuremath{\tilde f_{#1+}}\optspace}
\newcommand{\Pom}{{I\!P}}
\newcommand{\Reg}{{I\!R}}
% \newcommand{\xP}{x_\Pom}
\newcommand{\xP}{\xi}
\newcommand\sigRed{\ensuremath{\overline\sigma}}
\newcommand\DX{\ensuremath{\mathcal{X}}}

%\parindent=0pt
%\parskip=4pt

%\graphicspath{{figs/}}

%==========================================
\subsubsection {Cross-section}

\beq
  \frac{d\sigma}{d\beta\,dQ^2\,d\xP\,dt}
=
  \frac{2\pi\alpha^2}{\beta Q^4}\,
    \left( 1 +  (1-y)^2 \right) \sigRed^{D(4)}(\beta,Q^2,\xP,t)
\label{Dxs}
\eeq
where the `reduced cross-section', \sigRed, is defined as
\beq
\label{eq:sigred}
\sigRed
 = F_2 - \frac{y^2}{1 +  (1-y)^2}\, \FL
 = \FT + \frac{2(1-y)}{1 +  (1-y)^2}\, \FL
\eeq
Nb. $\xi$ is denoted by $x_\Pom$ in the H1 and ZEUS papers.

The dimension of 
\(
F_k^{D(4)}(\beta,Q^2,\xP,t)\)
is $\GeV^{-2}$
and
thus the quantities integrated over $t$
\beq
F_k^{D(3)}(\beta,Q^2,\xP)
\equiv
\int_{t_{\rm min}}^{t_{\rm max}} dt
F_k^{D(4)}(\beta,Q^2,\xP,t)
\eeq
are dimensionless.

Maximum kinematically allowed value of $t$ reads
\begin{equation}
t_{\rm MAX} 
=
-\frac{\xP^2 m_p^2 + p_\perp^2}{1-\xP}
\approx 
-\frac{\xP^2}{1-\xP} m_p^2
\end{equation}
where $m_p$ is the proton mass.

As $x = \xP\beta$ we can normalize to the standard DIS formula
\begin{equation}
\frac{d\sigma}{d\beta\,dQ^2\,d\xP\,dt} =
  \frac{2\pi\alpha^2}{x\, Q^4}\,
    \left( 1 +  (1-y)^2 \right) \xP\sigRed^{D(4)}(\beta,Q^2,\xP,t)
\end{equation}
which upon integration over $t$ reads
\begin{equation}
\label{Dxs3}
  \frac{d\sigma}{d\beta\,dQ^2\,d\xP}
=  
  \frac{2\pi\alpha^2}{x Q^4}\,
    \left( 1 +  (1-y)^2 \right) \,\xi\sigRed^{D(3)}(\beta,Q^2,\xP)
\end{equation}


The H1 and ZEUS data files typically contain $\xP\sigRed^{D(3)}$.

%==========================================
\subsubsection {Regge factorization}

For better data description we include a contribution from a secondary Reggeon, $\Reg$,
\beq
F_k^{D(4)}(\beta,Q^2,\xP,t) = 
\sum_{\mathcal{X} =\Pom,\Reg}
\phi_\mathcal{X}(\xP,t)\, F^\mathcal{X}_k(\beta,Q^2)
\eeq

or
\beq
\label{eq:FD3}
F_k^{D(3)}(\beta,Q^2,\xP) = 
\sum_{\mathcal{X} =\Pom,\Reg}
\Phi_\mathcal{X}(\xP)\, F^\mathcal{X}_k(\beta,Q^2)
\eeq
where
\begin{equation}
\label{eq:intFlux}
\Phi_{\mathcal{X}}(\xP) =
\int\limits_{t_{\rm min}}^{t_{\rm max}} dt\, \phi_\mathcal{X}(\xP,t)
\,.
\end{equation}

Parametrization of the fluxes
\begin{subequations}
\label{eq:flux}
\begin{equation}
\phi_\mathcal{X}(\xP,t) = 
\frac {A_\mathcal{X}\, e^{b_\mathcal{X} t}} {\xP^{2\alpha_\mathcal{X}(t) -1}}
\end{equation}
where
\begin{equation}
\alpha_\mathcal{X}(t) = \alpha_\mathcal{X}(0) + \alpha_\mathcal{X}' t
\,.
\end{equation}
\end{subequations}

$F^\Reg_k(\beta,Q^2)$ are taken as those of the pion.
%  with the normalization factor being absorbed in $\phi_\Reg(\xP,t)$.


%==========================================
%\subsubsection {Pomeron parametrization}

%The Pomeron is parametrized at the initial
%$Q_0^2$ in terms of two singlet distributions,
%$f_{g}$ and $f_{+}$.
%\begin{subequations}
%\label{singlet}
%\begin{eqnarray}
%\frac{d}{dt}f_{+} &=&
%\asotp\left[
%\AP_{\rm FF} f_{+} +\AP_{\rm FG}f_g
%\right]
%\\
%\frac{d}{dt}f_{g} &=&
%\asotp\left[
%\AP_{\rm GF} f_{+} +\AP_{\rm GG}f_g
%\right]
%\end{eqnarray}
%\end{subequations}

%As $\Pom$ is neutral, $f_{q} = f_{\bar q}$ for each flavour $q$.
%Assuming that all light quark PDFs are equal
%\begin{equation}
%f_d = f_u = f_s
%\,,
%\end{equation}
%we have
%\begin{subequations}
%\label{eq:pm}
%\begin{eqnarray}
%f_{q-} &\equiv& 0
%\\
%f_{q+} &\equiv& 2 f_q
%\end{eqnarray}
%\end{subequations}
%
%At \NF = 3
%\begin{equation}
%\label{eq:fq3}
%f_{q+} = f_{+}/3,\; q = d,u,s
%\,.
%\end{equation}
%% \ifFullVer
%\ie
%% \begin{equation}
%% \Sgl q = 0,\; q = d,u,s
%% \,,
%% \end{equation}
%% where
%\begin{equation}
%\Sgl q \equiv f_{q+} - \frac{1}{\NF}f_{+}
% = 0,\;\mbox{for}\; q = d,u,s
%\,.
%\end{equation}
% \fi
%
%This gives all PDFs for the FFNS, while for VFNS 
%$f_{h+}$ for $h=c,b,t$ are generated dynamically above the respective
%transition scales $Q_h^2$.
%Hence at $\NF > 3$ the singlet has contributions from the heavy quarks
%and we get non-trivial nonsinglet distributions $\Sgl{h}$ satisfying
%\begin{equation}
%\label{eq:nsevol}
%\frac{d}{dt}\Sgl h = \asotp\, \AP_{(+)} \Sgl h
%\end{equation}
%
%%\subsection {Parametrization at \texorpdfstring{$Q_0^2$}{Q0}}
%%\subsubsection {Parametrization at {$Q_0^2$}}
%{\bf Parametrization at {$Q_0^2$}} \\
%%-----------------------------------------------------------
%\label{sec:Par}
%
%Full PDFs are given in analogy to \Eq{eq:FD3}
%\begin{equation}
%f_k^{D(3)}(\beta,Q^2,\xP) =
%\hat\Phi_\Pom(\xP)\, f^\Pom_k(\beta,Q^2)
%+
%\Phi_\Reg(\xP)\, f^\Reg_k(\beta,Q^2)
%\end{equation}
%where $\hat\Phi_\Pom \equiv \Phi_\Pom/A_\Pom$,
%with the fluxes given by \Eq{eq:intFlux} and \Eq{eq:flux}.
%
%The Pomeron PDFs are parametrized as
%\def\Cini#1#2{A^{(#1)}_#2}
%\begin{equation}
%\label{eq:fP0}
%f^\Pom_N = \Cini N1  x^{\Cini N2} (1-x)^{\Cini N3}
%%   \left(1 + \Cini N4 x \right)
%  \; \exp\left(-\frac{d}{1.00001-x}\right)
%\,,
%\end{equation}
%where the `dumping factor' $d$ is taken as 0.01 or 0.001.
%$N = \mathrm G$ for gluon and $N = \mathrm S$ for `singlet': $f_{\rm S} \equiv f_+(\NF=3)$,
%\cf \Eq{eq:fq3}.

% The Reggeon PDFs $f^\Reg_k$ are taken from pion.

%%\subsubsection {HERAFitter parameters}
%{\bf HERAFitter parameters} \\
%%-----------------------------------------------------------
%\label{sec:HFitterPar}
%
%% minuit.in.txt
%% ExtraMinimisationParameters
%
%\begin{tabular}{l|l|l}
%Parameter & HERAFitter name & input file\\
%% \hline
%$\Cini {\mathrm G}1$ & Ag & minuit.in.txt \\
%$\Cini {\mathrm G}2$ & Bg & minuit.in.txt \\
%$\Cini {\mathrm G}3$ & Cg & minuit.in.txt \\
%$\Cini {\mathrm S}1$ & Auv & minuit.in.txt \\
%$\Cini {\mathrm S}2$ & Buv & minuit.in.txt \\
%$\Cini {\mathrm S}3$ & Cuv & minuit.in.txt \\
%$\alpha_\Pom(0)$ & Pomeron\_a0 & steering.txt \\
%$A_\Reg$ & Reggeon\_factor & steering.txt \\
%$\alpha_\Reg(0)$ & Reggeon\_a0 & steering.txt \\
%\end{tabular}

\endinput


