
\label{sec:install}
%%%%%%%%%%%

The Installation Instructions are dependent on which modules are activated via the configuration option. 
\subsubsection{Pre-requirements}

The following packages are needed in order to build \fitter\ package:
\begin{itemize}
\item QCDNUM~\cite{qcdnum} version at least {\tt qcdnum-17-00-04}, can be found at \\
  {\tt http://mbotje.web.cern.ch/mbotje/qcdnum/Site/QCDNUM17.html}
\item {\tt CERNLIB} libraries. Note that for {\tt CERNLIB} one can use {\tt /afs/} installation from CERN:\\
  {\tt /afs/cern.ch/sw/lcg/external/cernlib/}
%\item Link to recent Root libraries (e.g. version 5.26)
%\item Optional: {\tt APPLGRID}
\end{itemize}
The \fitter\ program has been tested on various platforms:\\
SL4, SL5 (32 and 64 bit), SL6 (64 bit),  Ubuntu 10.10 (gcc and gfortran version 4.6.3), Mac OS (gcc and gfortran version 4.7.2) 

%%%%%%%%%%%
\subsubsection{Default Installation}
\begin{itemize}
\item
 Specify {\tt CERN\_ROOT} 
     and {\tt QCDNUM\_ROOT} variables such that\\
     \verb'$CERN_ROOT/lib'  and \verb'$QCDNUM_ROOT/lib'
 point to the corresponding libraries
\item Run:
\begin{verbatim}
    ./configure
    make 
    make install
\end{verbatim}
After these commands are finished, the executable {\tt bin/FitPDF} 
file should be installed
\item  Run a check:
\begin{verbatim}
    bin/FitPDF 
\end{verbatim}
\end{itemize}
%%%%%%%%%%%
\subsubsection{Installation with {\tt APPLGRID}}
\begin{itemize}
\item
 Specify {\tt CERN\_ROOT} and {QCDNUM\_ROOT} variables such that\\
     \verb'$CERN_ROOT/lib' and \verb'$QCDNUM_ROOT/lib'
 point to the corresponding libraries
\item Make sure that {\tt \$PATH} and {\tt \$LD\_LIBRARY\_PATH} 
variables point to the {\tt APPLGRID} environment.
\item Run:
\begin{verbatim}
    ./configure --enable-applgrid
    make 
    make install
\end{verbatim}
After these commands are finished, the executable {\tt bin/FitPDF} 
file should be installed
\item  Run a check:
\begin{verbatim}
    bin/FitPDF 
\end{verbatim}
\end{itemize}
%%%%%%%%%%%
\subsubsection{Installation with {\tt LHAPDF}}\label{sec:install_lhapdf}

Installation with LHAPDF requires the {\tt LHAPDF} package, available online at:\\
{\tt http://lhapdf.hepforge.org/install}.
Then
\begin{verbatim}
tar -xvzf lhapdf-v.r.p.tar.gz
cd lhapdf-v.r.p
./configure --prefix=/path/to/directory (and/or --enable-low-memory)
make
make install
cd /path/to/directory/share/lhapdf
mkdir PDFsets
\end{verbatim}

Once installed, create the path that will be linked to the {\tt HERAFitter} package.\\
 Specify \verb'LD_LIBRARY_PATH'
     and {\tt LHAPATH} variables such that they
 point to the corresponding libraries, and PDF sets location (where lhapdf tables are stored)
\begin{verbatim}
export LD_LIBRARY_PATH=/path/to/directory/lhapdf-v.r.p/lib:$LD_LIBRARY_PATH
export LHAPATH=/path/to/directory/share/lhapdf/PDFsets
\end{verbatim}



%%%%%%%%%%%
\subsection{Installation with {\tt PDF reweighting}}\label{sec:install_nnpdfrweight}

Note: For installation allowing for PDF reweighting, the latest version of {\tt LHAPDF}, lhapdf-5.8.7b2, should be installed.

\begin{itemize}
\item Make sure that {\tt \$LD\_LIBRARY\_PATH} includes the LHAPDF libraries.
\item Run:
\begin{verbatim}
    ./configure --enable-lhapdf  --enable-nnpdfWeight
    make 
    make install
\end{verbatim}
After these commands are finished, the executable {\tt bin/FitPDF} 
file should be installed
\item Set {\tt FLAGRW = True} in the steering file and change also the other parameters of the {\tt \&reweighting} namelist if needed.
\item  Run a check:
\begin{verbatim}
    bin/FitPDF 
\end{verbatim}
\end{itemize}


%%%%%%%%%%%
\subsubsection{Installation with {\tt HATHOR}}

 \begin{itemize}
  \item Download Hathor from 
\begin{verbatim}
http://www-zeuthen.desy.de/~moch/hathor/
\end{verbatim}
     and install it according to the instructions given there
     (requires \verb'LHAPDF' library)

  \item Define a variable HATHOR\_ROOT  such that HATHOR\_ROOT  points to the
     directory of your Hathor installation

  \item Install the HERAFitter as described above but configuring it
     with the option "--enable-hathor" before building it
 \end{itemize}


%%%%%%%%%%%
%\subsection{Installation with {\tt CASCADE}}
\subsubsection{Installation for TMD (uPDF) in high-energy factorisation (using  {\tt CASCADE})}

\begin{itemize}

\item Installation with TMD requires Cascade and Pythia generators, 
they can be downloaded from\\
{\tt http://cascade.hepforge.org/ } and 
{\tt https://pythia6.hepforge.org/ } respectively. \\
        \\
After installation of generator packages, the {\tt CASCADE\_ROOT}  and {\tt PYTHIA\_ROOT} 
environment variables have be specified and point to the corresponding libraries. 
In DESY afs enviroment the pre-installed versions of Cascade and Pythia can be used:  
%
{\footnotesize\begin{verbatim}
export CASCADE\_ROOT=/afs/desy.de/group/alliance/mcg/public/MCGenerators/cascade/2.2.04/\$SYSNAME 
export PYTHIA\_ROOT=/afs/desy.de/group/alliance/mcg/public/MCGenerators/pythia6/425/\$SYSNAME}
\end{verbatim} }
\normalsize
where {\tt SYSNAME } = i586\_rhel50 or similar.

\item Run:
\begin{verbatim}
    ./configure --enable-uPDF
    make 
    make install
\end{verbatim}


\item use steering and minuit input files from "input\_steering": 

   \begin{verbatim} 
   cp input-steering/steering.txt.kt-factorisation steering.txt 
   cp input-steering/minuit.in.txt.kt-factorisation minuit.in.txt 
   cp input-steering/steer-ep-CASCADE steer-ep 
   \end{verbatim}

\item  edit steering.txt: 
   \begin{verbatim}
   \&CCFMFiles: give name for output grid file for uPDF   
   \&HERAFitter 
   TheoryType = 'uPDF3' ! 'DGLAP'  -- colinear evolution
                             ! 'DIPOLE' -- dipole model 
                             ! 'uPDF'   -- un-integrated PDFs:
                             !uPDF1 fit with kernel ccfm-grid.dat file
                             !uPDF2 fit evolved uPDF, fit just normalisation
                             !uPDF3 fit using precalculated grid of sigma_hat
  \end{verbatim}

\item run the program: bin/FitPDF 
   
\item plotting $F_2$ fit results: \\
DrawResults  will draw $F_2$ results. \\
The uPDFs need to be plotted with an external package (currently not available).
\end{itemize}

