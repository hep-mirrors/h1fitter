\documentclass[11pt,a4paper]{article}
\usepackage{graphicx,epsfig}
\usepackage{hhline}

\usepackage{amsmath,amssymb}
\usepackage{times}
\usepackage[varg]{txfonts}
\DeclareMathAlphabet{\mathbold}{OML}{txr}{b}{it}

\usepackage{array,multirow,dcolumn}
%\usepackage[mathlines,displaymath]{lineno}
\usepackage{rotating}

 
% we use natbib instead of cite to work with hyperref
%\usepackage{cite}
%\usepackage[numbers,square,comma,sort&compress]{natbib}
%\usepackage{hypernat}
\usepackage{textcomp}

\bibliographystyle{alpha}
\renewcommand{\topfraction}{1.0}
\renewcommand{\bottomfraction}{1.0}
\renewcommand{\textfraction}{0.0}

% \renewcommand{\arraystretch}{1.2}
\newlength{\dinwidth}
\newlength{\dinmargin}
\setlength{\dinwidth}{21.0cm}
\textheight24cm \textwidth16.0cm
\setlength{\dinmargin}{\dinwidth}
\setlength{\unitlength}{1mm}
\addtolength{\dinmargin}{-\textwidth}
\setlength{\dinmargin}{0.5\dinmargin}
\oddsidemargin -1.0in
\addtolength{\oddsidemargin}{\dinmargin}
\setlength{\evensidemargin}{\oddsidemargin}
\setlength{\marginparwidth}{0.9\dinmargin}
\marginparsep 8pt \marginparpush 5pt
\topmargin -42pt
\headheight 12pt
\headsep 30pt \footskip 32pt
\parskip 3mm plus 2mm minus 2mm


\newcommand\fitter{ \mbox{\tt HERAFitter} }
\title{\fitter\ - PDF Fitting package}
%\author{H1 Collaboration}
\begin{document}
\maketitle
\begin{abstract}
\end{abstract}
\tableofcontents
\newpage
%%%%%%%%%%%%%%%%%%%%%%%%%%%%%%
\section{Introduction}
This manual provides a short description of the \fitter\ program 
which can be used to determine unpolarised parton density functions 
(PDFs) using deep inelastic scattering (DIS) data and other processes such as 
Drell-Yan, jet or ttbar processes.
The parton density functions are needed to calculate cross sections
for the $ep$ and $pp$ colliders and thus required for interpetation
of the data collected at the LHC.
% The \fitter\ program were used to determine the HERA1.0 PDF set~\cite{h1zeus:2009wt}.

The manual begins with a brief discussion of the theoretical calculation
used in the program (section~\ref{sec:theory}) followed by description of the
PDF parameterisation (section~\ref{sec:pdfparam}) and various $\chi^2$ functions used in the
minimisation (section~\ref{sec:chi2}). The installation instructions are given in
section~\ref{sec:install}, depending on various configuration options chose. A description of the program steering cards and
the output options is given in section~\ref{sec:man}.
%%%%%%%%%%%%%%%%%%%%%%%%%%%%%%
\section{Program Installation Instructions} 
\label{sec:install}
%%%%%%%%%%%

The Installation Instructions are dependent on which modules are activated via the configuration option. 
\subsection{Pre-requirements}

The following packages are needed in order to build \fitter\ package:
\begin{itemize}
\item QCDNUM~\cite{qcdnum} version at least {\tt qcdnum-17-00/04}, can be found at \\
  {\tt http://mbotje.web.cern.ch/mbotje/qcdnum/Site/QCDNUM17.html}
\item {\tt CERNLIB} libraries. Note that for {\tt CERNLIB} one can use {\tt /afs/} installation from CERN:
  {\tt /afs/cern.ch/sw/lcg/external/cernlib/}
%\item Link to recent Root libraries (e.g. version 5.26)
%\item Optional: {\tt APPLGRID}
\end{itemize}
The \fitter\ program has been tested on various platforms: 
   SL4, SL5 (32 and 64 bit),  Ubuntu 10.10.
%%%%%%%%%%%
\subsection{Default Installation}
\begin{itemize}
\item
 Specify {\tt CERN\_ROOT} 
     and {\tt QCDNUM\_ROOT} variables such that 
     {\tt \$CERN\_ROOT/lib}  and {\tt \$QCDNUM\_ROOT/lib}
 point to the corresponding libraries
\item Run:
\begin{verbatim}
%    autoreconf --install
    ./configure
    make 
    make install
\end{verbatim}
After these commands are finished, the executable {\tt bin/FitPDF} 
file should be installed
\item  Run a check:
\begin{verbatim}
    bin/FitPDF 
\end{verbatim}
\end{itemize}
%%%%%%%%%%%
\subsection{Installation with {\tt APPLGRID}}
\begin{itemize}
\item
 Specify {\tt CERN\_ROOT} and {QCDNUM\_ROOT} variables such that 
     {\tt \$CERN\_ROOT/lib}  and {\tt \$QCDNUM\_ROOT/lib}
 point to the corresponding libraries
\item Make sure that {\tt \$PATH} and {\tt \$LD\_LIBRARY\_PATH} 
variables point to the {\tt APPLGRID} environment.
\item Run:
\begin{verbatim}
    autoreconf --install
    ./configure --enable-applgrid
    make 
    make install
\end{verbatim}
After these commands are finished, the executable {\tt bin/FitPDF} 
file should be installed
\item  Run a check:
\begin{verbatim}
    bin/FitPDF 
\end{verbatim}
\end{itemize}
%%%%%%%%%%%
\subsection{Installation with {\tt LHAPDF}}\label{sec:install_lhapdf}

Installation with LHAPDF requires the {\tt LHAPDF} package, available online at:\\
{\tt http://lhapdf.hepforge.org/install}.
Then
\begin{verbatim}
tar -xvzf lhapdf-v.r.p.tar.gz
cd lhapdf-v.r.p
./configure --prefix=/path/to/directory (and/or --enable-low-memory)
make
make install
cd /path/to/directory/share/lhapdf
mkdir PDFsets
\end{verbatim}

Once installed, then create the path that will be linked to the {\tt HERAFitter} package.
 Specify {\tt LD\_LIBRARY\_PATH} 
     and {\tt LHAPATH} variables such that they
 point to the corresponding libraries, and PDF sets location (where lhapdf tables are stored)
\begin{verbatim}
export LD_LIBRARY_PATH=/path/to/directory/lhapdf-v.r.p/lib:\$LD_LIBRARY_PATH
export LHAPATH=/path/to/directory/share/lhapdf/PDFsets
\end{verbatim}



%%%%%%%%%%%
\subsection{Installation with {\tt PDF reweighting}}\label{sec:install_nnpdfrweight}

Note: For installation allowing for PDF reweighting, the latest version of {\tt LHAPDF}, lhapdf-5.8.7b2, should be installed.

\begin{itemize}
\item Make sure that {\tt \$LD\_LIBRARY\_PATH} includes the LHAPDF libraries.
\item Run:
\begin{verbatim}
    autoreconf --install
    ./configure --enable-lhapdf  --enable-nnpdfWeight
    make 
    make install
\end{verbatim}
After these commands are finished, the executable {\tt bin/FitPDF} 
file should be installed
\item Set {\tt FLAGRW = True} in the steering file and change also the other parameters of the {\tt \&reweighting} namelist if needed.
\item  Run a check:
\begin{verbatim}
    bin/FitPDF 
\end{verbatim}
\end{itemize}


%%%%%%%%%%%
\subsection{Installation with {\tt HATHOR}}

 \begin{itemize}
  \item Download Hathor from 
\begin{verbatim}
http://www-zeuthen.desy.de/~moch/hathor/
\end{verbatim}
     and install it according to the instructions given there
     (requires \begin{verbatim}LHAPDF \end{verbatim}\ library)

  \item Define a variable HATHOR\_ROOT  such that HATHOR\_ROOT  points to the
     directory of your Hathor installation

  \item Install the H1Fitter as described above but configuring it
     with the option "--enable-hathor" before building it
 \end{itemize}


%%%%%%%%%%%
\subsection{Installation with {\tt CASCADE}}

\begin{itemize}

\item  set environment variables (with {SYSNAME=i586\_rhel50} or similar)
    
\begin{verbatim}
    export CERN\_ROOT=/cern/pro  
    export QCDNUM\_ROOT=/h1wgs/h1desy11/x04/usr/glazov/openfitter/qcdnum-17-00-03
    export CASCADE\_ROOT=/afs/desy.de/group/alliance/mcg/public/MCGenerators/cascade/2.2.04/\$SYSNAME 
    export PYTHIA\_ROOT=/afs/desy.de/group/alliance/mcg/public/MCGenerators/pythia6/425/\$SYSNAME}
    \end{verbatim}	

\item use steering and minuit input files from "input\_steering": 

   \begin{verbatim} 
   cp input-steering/steering.txt.kt-factorisation steering.txt 
   cp input-steering/minuit.in.txt.kt-factorisation minuit.in.txt 
   cp input-steering/steer-ep-CASCADE steer-ep 
   \end{verbatim}

\item  edit steering.txt: 
 \begin{verbatim}
   \&CCFMFiles: give name for output grid file for uPDF.\&H1Fitter 
   \&H1Fitter \\ 
   \!  ITheory = 101   ! =101 fit with kernel ccfm-grid.dat file 
   \!  ITheory = 102   ! =102 fit evolved uPDF, fit just normalisation 
    ITheory = 103   ! =103 fit using precalculated grid of $sigma\_hat$
       all other parameters are standard
  \end{verbatim}

\item run the program: bin/FitPDF 
   
\item plotting F2 fit results:
   DrawResults \! will draw F2 results to plot uPDF one can use the "updfplotter" web interface $\to$ copy updf grid (with name from "CCFMFiles" under 2 to somewhere .... this description needs improvements
\end{itemize}

%%%%%%%%%%%%%%%%%%%%%%%%%%%%%%
\section{Theoretical Input}
\label{sec:theory}
The \fitter\ program uses currently as standard the DGLAP~\cite{Gribov:1972ri,Gribov:1972rt,Lipatov:1974qm,Dokshitzer:1977sg,Altarelli:1977zs}
 evolution equations as implemented in the QCDNUM~\cite{qcdnum} program. The fit 
procedure begins with parameterising the input PDFs at the starting 
scale $Q^2_0$ which should be chosen to be below the charm mass threshold
$m_C^2$.
The PDFs are then evolved using the DGLAP evolution equations  
at NLO~\cite{Curci:1980uw,Furmanski:1980cm} in the $\overline{MS}$ scheme.
The renormalisation and factorisation scales are set to $Q^2$. The \fitter\ program
also allows for LO and NNLO evolution. 

The cross-section predictions are obtain by convoluting the PDFs with the 
hard scattering coefficient functions. For the DIS processes, those are calculated 
using the general mass variable-flavour scheme. 
The program implements the  zero mass scheme from QCDNUM as well as
various treatments for the heavy quark thresholds as provided by the MSTW group
- the RT scheme with its variants at NLO and NNLO ~\cite{Thorne:1997ga,Thorne:2006qt}, as provided by the CTEQ group - the ACOT scheme with its variants at LO and NLO, as provided by the ABM group - the BMSN scheme at NLO and NNLO.
Each of these schemes is briefly discussed in further details.

The jet cross sections
are calculated using APPLGRID and FastNLO. The program has two implementations
for $pp$  DY processes. The first implementation uses
calculations at LO which can be extended to NLO using k-factors,
the second uses the APPLGRID interface.
For thorough details of the theoretical modules we direct the user to read the provided  references of these packages.


%%%%%%%%%%%
\subsection{DIS and Schemes}
There are different approaches to the treatment of the heavy quark production. These include the Fixed Flavour Number (FFN) and Variable Flavour Number (VFN) schemes.
%%%%
\subsubsection{ZMVFNS}
The evolution program QCDNUM~\cite{qcdnum} used in \fitter\ provides 
the calculation of the deep inelastic structure functions in the zero-mass, 
generalised mass and the fixed flavour number schemes. 
In the zero-mass variable flavour number scheme (ZM-VFNS) heavy quark densities are
included into proton above quark masses but they are treated as massless in both,
the initial and final states.
This scheme is accurate in the region where $Q^2$ is so much greater than $m_h^2$
but becomes unreliable for $Q^2 \sim m_h^2$. \\
The un-polarised DIS structure functions in ZM-VFN scheme are computed as a 
convolution of the parton densities with zero-mass coefficient functions and 
in \fitter\ are activated via namelist {\tt HF$\_$SCHEME} in the {\tt steering.txt}.
%%%%
\subsubsection{TR}

The Thorne-Roberts (TR) scheme is a general-mass variable flavour number scheme (GM-VFNS) used as default for the MTSW PDF sets. GM-VFNS smoothly connect the two reagions: scale below the heavy quark threshold and the scale above the heavy quark threshold. However, the definition is not unique.
A GM-VFNS can be defined by demanding equivalence of the nf = n (FFNS) and nf = n+1 flavour (ZM-VFNS) descriptions above the transition point for the new parton distributions
(they are by definition identical below this point), at all orders. However, the equivalency of swapping te O(mH2/Q2) terms without violating the definition of a GM-VFNS is what mainly distinguish the ACOT from TR schemes. 
One major issue in a complete GM-VFNS, is that of the ordering of the
perturbative expansion. This ambiguity comes about because the ordering in alphas is different for the number of active flavours.


 
%%%%
\subsubsection{ACOT}

The Aivazis-Collins-Olness-Tung scheme  belongs to the group of VFN factorization schemes that uses the renormalization method of Collins-Wilczek-Zee (CWZ).This scheme involves a mixture of the $\bar{MS}$ scheme for light partons (and for heavy partons if the factorisation scale is larger than the heavy mass)and the zero-momentum subtraction renormalization scheme for graphs with heavy quark lines (if factorisation scale is smaller than the mass of the heavy quark threshold).

There are different variants of this scheme which are all incorporated in the HERAFitter framework.

%%%%
\subsubsection{ABM}
An interface to open-source code OPENQCDRAD~\cite{openqcdrad:page} in HERAFitter framework 
provides an access to 
fixed flavour number scheme (FFNS)~\cite{Laenen:1992,Laenen:1993,Riem:1995}~\footnote{Besides 
the variable flavour number 
scheme, QCDNUM also supports the fixed flavour number scheme which can be used for variouse 
cross checks with OPENQCDRAD.}. In FFNS the number of light quark 
flavours $n_{f}$ (here $n_{f}=3$) are considered in the PDF evolution and heavy (massive) 
quarks appear only in the final state. 
The QCD corrections to the massive Wilson coefficients which are known up to the NNLO
for the neutral-current (NC) heavy-quark production~\cite{} are implemented in OPENQCDRAD.
In the case of charged-current (CC), the massive NLO QCD corrections~\cite{} are available.
In addition, the treatment of the heavy-quark contributions in DIS are provided 
in both, the pole-mass and the running-mass definition in $\overline{\text{MS}}$ 
scheme~\cite{Alekhin:runm}. \\
In case the FFN scheme is chosen as the fitting option (see corresponding instructions in the section 1),
the heavy quark contributions to DIS structure functions $F_2$ and $F_L$ (and $F_3$ in the charged 
current case) are calculated in the FFNS and together with the light-flavor contributions are 
provided for the theory prediction calculation (theory$\_$dispatcher.f).
The interpolation to PDFs and $\alpha_s$ evolution from QCDNUM are set up in the interface to OPENQCDRAD. \\
The variation of the renormalisation and factorisation scales for heavy quarks is 
possible (see available options in steering.txt file).
%
%       
%%%%%%%%%%%
\subsubsection{$ep$ Electroweak corrections via HS}
%%%%%%%%%%%
\subsection{DY process}
%%%%
\subsubsection{Leading Order}
The leading order Drell-Yan~\cite{Drell:1970wh,Yamada:1981mw} cross section 
for the neutral current, triple differential in
invariant mass \(M\), boson rapidity \(y\) and CMS
lepton scattering angle \(\cos\theta\), can be written as
\begin{align}
\frac{\mathrm{d}^3\sigma}{\mathrm{d}M\mathrm{d}y\mathrm{d}\cos\theta} &=
  \frac{\pi\alpha^2}{3MS}\sum_{q}P_q
  \left[F_q(x_1,Q^2)F_{\bar{q}}(x_2,Q^2) + (q\leftrightarrow\bar{q})\right],
\end{align}
where \(S\) is a squared CMS beam energy, \(x_{1,2} = \frac{M}{\sqrt{S}}\exp(\pm y)\) and 
\begin{align}
  P_q &=  e_l^2e_q^2(1+\cos^2\theta) \nonumber \\
      &+  e_le_q\frac{2M^2(M^2-M_Z^2)}{\sin^2\theta_W\cos^2\theta_W
          \big[(M^2-M_Z^2)^2+\Gamma_Z^2M_Z^2\big]}
          \big[aA_q(1+\cos^2\theta)+2bB_q\cos\theta\big] \nonumber \\
      &+  \frac{M^4}{\sin^4\theta_W\cos^4\theta_W
          \big[(M^2-M_Z^2)^2+\Gamma_Z^2M_Z^2\big]}
          \big[(a^2+b^2)(A_q^2+B_q^2)(1+\cos^2\theta)+8abA_qB_q\cos\theta\big].
\end{align}
Here \(\theta_W\) is the Weinberg angle, \(M_Z\) and \(\Gamma_Z\) are Z boson mass and 
width, and
\begin{align}
 a & = -\frac{1}{4} + \sin^2\theta_W,  \nonumber \\
 b & = -\frac{1}{4},  \nonumber \\
 A_q & = \frac{1}{2}I_q^3-e_q\sin^2\theta_W, \nonumber \\
 B_q & = \frac{1}{2}I_q^3,  \nonumber \\
 I_u^3 & = -I_d^3 = \frac{1}{2},  \nonumber \\
 e_l & = -1, e_u = \frac{2}{3}, e_d = -\frac{1}{3}.
\end{align}

The expression for charged current has simpler form:
\begin{align}
\frac{\mathrm{d}^3\sigma}{\mathrm{d}M\mathrm{d}y\mathrm{d}\cos\theta} &=
 \frac{\pi\alpha^2}{48S\sin^4\theta_W}
 \frac{M^3(1-\cos\theta)^2}{(M^2-M_W^2)+\Gamma_W^2M_W^2}
 \sum_{q_1,q_2}V_{q_1q_2}^2F_{q_1}(x_1,Q^2)F_{q_2}(x_2,Q^2),
\end{align}
where \(V_{q_1q_2}\) is the CKM quark mixing matrix and \(M_W\) and \(\Gamma_W\)
are \(W\) boson mass and decay width.

The simple form of these expressions allows to calculate integrated
cross sections without utilization of Monte-Carlo techniques.
This is particularly useful for PDF fitting purposes because
the statistical fluctuations are avoided in this case. In both 
neutral and charge current expressions the parton density functions
factorize as a function dependent only on boson rapidity \(y\) and
invariant mass \(M\) leaving \(\cos\theta\) dependence aside.
The integral in \(\cos\theta\) can be computed analytically and
integrations in \(y\) and \(M\) can be performed with Simpson
method. The \(\cos\theta\) parts are kept in the equation 
explicitly because their integration is asymmetric for
data in lepton \(\eta\) bins and also is being performed when applying 
the lepton \(p_{\perp}\) cuts.

The fact that PDF functions factorize with the rest part of 
expression allows to significantly boost calculations when 
performing parameter fits over lepton rapidity data. In this case
the factorized part of expression independent on PDFs can be
calculated only once for all minimization iterations.
The leading order code in HERAFitter package implements this 
optimization and uses fast convolution routines provided by
QCDNUM. Currently the full width LO calculations are optimized 
for lepton pseudorapidity and boson rapidity distributions with
possibility to apply lepton \(p_{\perp}\) cuts.

The calculated leading order cross sections are multiplied by
NLO or NNLO K-factors provided for corresponding data distributions.
%%%%
\subsubsection{Next Leading Order}
%%%%%%%%%%%
\subsection{$t\bar{t}$ Cross Sections via {\tt HATHOR}}
Top-quark pairs ($t\bar{t}$) are mainly produced via $gg$ fusion and
$q \bar q$ annihilation. Furthermore, there are the $q q'$ and
$q g$ production modes.
The program HATHOR~\cite{Aliev:2010zk} allows calculating
the expected total $t \bar t$ cross section at hadron colliders
($p \bar p$ and $p p$) up to approximate NNLO accuracy.
Version 1.3 of HATHOR includes the exact NNLO for $q \bar q \to t \bar t$ \cite{Baernreuther:2012ws}
as well as a new high-energy constraint on the approximate NNLO obtained from
soft-gluon resummation \cite{Moch:2012mk}.
The default choice for renormalization and factorization scale in $t \bar t$ production is the top-quark mass, $m_t$.
The pole mass scheme is typically employed for $m_t$ but HATHOR also supports calculations in
the $\overline{\text{MS}}$ scheme.
\subsection{Jets}
The calculation of higher order jet cross sections is very demanding
in means of computing power. The reasons are the large number of contributing
Feynman diagrams and also the large number of infrared divergencies.
For an accurate cancellation of these singularities, typically, the 
dipole subtraction method is applied in such calculations.
During the necessary Monte Carlo integration a very fine phase
space sampling has to be performed in order to account for the
accurate cancellation of the counter terms.

In order to enable the inclusion of jet-cross section 
measurements in PDF and $\alpha_s$ fits, these perturbative
coefficients have to be pre-computed in a PDF and $\alpha_s$ 
independent way. For this purpose, two quite similar tools are
interfaced to the HERAFitter.

\subsubsection{FastNLO}
The fastNLO project~\cite{Kluge:2006xs,Wobisch:2011ij,Britzger:2012bs}
enables the inclusion of jet data in PDF and $\alpha_s$ fits.
This tool uses multi-dimensional interpolation
techniques to convert the convolutions of perturbative 
coeffcients with parton distribution functions and 
the strong coupling into simple products.
Although the concept is process independent, the perturbative 
coefficients are usually calculated by the \texttt{NLOJET++}
program~\cite{Nagy:1998bb} where calculations for jet-production
in DIS~\cite{Nagy:2001xb}  as well as in hadron-hadron 
collisions~\cite{Nagy:2003tz,Nagy:2001fj} are available.
Also threshold-corrections of $\mathcal{O}$(NNLO) for 
inclusive jet cross sections in hadron-hadron collisions are
available~\cite{Kidonakis:2000gi}.

The fastNLO libraries are standardized included in the HERAFitter
package and no further requirements or compilation options
are needed. In order to include a new measurement into the PDF-fit,
the fastNLO table have to be specified. These tables include all
necessary informations of the perturbative coefficients and the
calculated process for all bins of a certain dataset. 
Tables for almost all published jet measurements
are available through the project website {\tt http://fastnlo.hepforge.org},
or have otherwise to be calculated by using the full fastNLO package.

Features of the fastNLO concept are the very quick convolution of the
perturbative coefficients with the PDFs of
$\mathcal{O}(100 ms)$ and the very high accuracy
of the interpolation procedure. 
The fastNLO tables are conventionally calculated
for multiple factors of the factorization scale, 
and the renormalization scale factor can be choosen freely.
Some of the fastNLO tables already involve a scale-independent
concept~\cite{Britzger:2012bs}, which allows for 
the free choice of the renormalization and the factorization
scale as a function of two pre-defined observables.
The evaluation of the strong coupling constant, which enters
the cross section calculation, is taken consistently from the 
QCDNUM evolution code.


\subsubsection{APPLGRID}
The APPLGRID~\cite{Carli:2010rw} package allows to compute a fast estimate
of NLO cross section for particular processes for arbitrary set of 
proton parton density functions. The package implements
calculation of cross section of electroweak boson (\(Z,W\))
production as well as jet production in proton-(anti)proton
collisions and DIS processes. 

The approach is based on storing perturbative coefficients
of NLO QCD calculations of final-state observables measured
in hadron colliders in look-up tables. The PDFs and the 
strong couplings are included during the final calculations,
e.g. during the PDF fits procedure. The method allows 
variation of factorization and renormalization scales in
calculations.

The look-up tables (grids) can be generated with modified versions of
of MCFM~\cite{Campbell:1999ah,Campbell:2010ff} or 
NLOjet++~\cite{Nagy:2001fj} software distributed
with the full version of APPLGRID package. NLO calculations
for the current analysis are performed with the help of APPLGRID
generated grids based on MCFM calculations. 

Run parameters
and electroweak parameters are set in MCFM in a standard way
via the input file and user's part of the code. 
Binning and definitions of the observables for which the
differential cross sections are needed are set in the 
APPLGRID code. 
The grid parameters \(x_1, x_2\) and \(Q^2\) binning
and interpolation orders are also defined in the code.

APPLGRID performs construction of the look-up tables in two 
steps: {\it (i)} exploration of the phase space in order
to optimize the memory storage and {\it (ii)} actual grid
construction in the phase space corresponding to the 
requested observables.

Afterwards the NLO cross sections are restored from the grids
with providing PDFs, \(\alpha_S\), factorization and 
renormalization scales and with QCD NNLO k-factors applied
if stated.

%%%%%%%%%%%
\subsection{DIPOLE models}

At low $x$ and low $Q^{2}$, virtual photon-proton scattering is described using the colour
dipole model formalism~\cite{NNZ:91}. Within this formalism, the scattering process is calculated as a fluctuation of the
photon into a quark-antiquark pair (dipole), with a lifetime $\propto\enskip 1/x$, which interacts with the proton.

Several approaches have been developed to phenomenologically describe the dipole-proton interaction
cross section, three of which are implemented in the HERAFitter. These are
the original model version (GBW)~\cite{Golec-Biernat:1998js}, a model based on the colour glass condensate approach
to the high parton density regime (IIM)~\cite{Iancu:2003ge}, and a modified GBW model by adding effects of the 
DGLAP evolution (BGK)~\cite{Bartels:2002cj}.
%%%%
\subsubsection{GBW model}
In the GBW model the dipole-proton cross section $\sigma_{\text{dip}}$ is given by
\begin{equation}
\label{eGBW}
   \sigma_{\text{dip}}(x,r^{2}) = \sigma_{0} \left(1 - \exp \left[-\frac{r^{2}}{4R_{0}^{2}(x)} \right]\right),
\end{equation}
where $r$ corresponds to the transverse separation between the quark and the antiquark, and $R_{0}^{2}$ is 
an $x$ dependent scale parameter, having the form $R_{0}^{2}(x)=\left(x/x_{0}\right)^{\lambda}$.
The free fitted parameters are the cross-section normalisation $\sigma_{0}$ as well as $x_{0}$ and $\lambda$.
%%%%
\subsubsection{IIM model}
The IIM model assumes an improved expression for dipole cross section which is based on the 
Balitsky-Kovchegov equation~\cite{Balitsky:1995ub}. The explicit formula for $\sigma_{\text{dip}}$ 
can be find in~\cite{Iancu:2003ge}. The free fitted parameters are the alternative scale parameter $\tilde{R}$, $x_{0}$ and $\lambda$.
%%%%
\subsubsection{BGK model}
The BGK model modifies the equation (\ref{eGBW}) by incorporating the LO and NLO DGLAP evolution
of the gluon distribution. This leads to the expression for the dipole cross section
\begin{equation*}
\label{eBGK}
   \sigma_{\text{dip}}(x,r^{2}) = \sigma_{0} \left(1 - \exp \left[-\frac{\pi^{2} r^{2} \alpha_{s}(\mu^{2}) xg(x,\mu^{2})}{3 \sigma_{0}} \right]\right).
\end{equation*}
The factorization scale $\mu^{2}$ has the form $\mu^{2} = C_{bgk}/r^{2}+\mu^{2}_{0}$.
This model uses the following gluon density at the starting scale $Q_{0}^{2}=1\mbox{ GeV}^{2}$
\begin{equation*}
\label{eqTH730}
   xg(x,Q^{2}_{0}) = A_{g} x^{-\lambda_{g}}(1-x)^{C_{g}}.
\end{equation*}
The free fitted parameters for this model are $\sigma_{0}$, $\mu^{2}_{0}$ and 3 parameters for gluon $A_{g}$, $\lambda_{g}$, $C_{g}$. The parameter $C_{bgk}$ is kept fixed: $C_{bgk} = 4.0$. 
%%%%
\subsubsection{Mixed with DGLAP model}
%%%%%%%%%%%
\subsection{Unintegrated PDFs using CASCADE}
%%%%%%%%%%%
\subsection{Diffractive DIS PDFs}
Diffractive DIS data are fitted within the 'proton vertex factorisation' approach where 
the diffractive DIS is mediated by the exchange of hard Pomeron and a secondary Reggeon.
The model supplied by the DiffDIS package provides values of the 'reduced cross section',
$\sigma_r = F_2 - y^2/(1+(1-y)^2) F_L$
which is expected to be the experimentally meausured quantity.
The model supplied by the DiffDIS package provides values of the 'reduced cross section',
$\sigma_r = F_2 - y^2/(1+(1-y)^2) F_L$
which is expected to be the experimentally meausured quantity.
\\
\newcommand\Version{2.01.02}
\newif\ifFullVer\FullVertrue
\FullVerfalse

% \texorpdfstring

\renewcommand\topfraction{0.5}
\renewcommand\bottomfraction{0.5}
\renewcommand\textfraction{0.5}
\renewcommand\floatpagefraction{0.5}

\makeatletter
\def\comsp{\@ifnextchar,\relax{\@ifnextchar\ \relax{\@ifnextchar:\relax{\@ifnextchar.\relax\ }}}}
\makeatother
\DeclareRobustCommand{\ie}{{\it i.e.}\comsp}
\DeclareRobustCommand{\eg}{{\it e.g.}\comsp}
\DeclareRobustCommand{\cf}{{\it cf.}\comsp}
\DeclareRobustCommand{\etal}{{\it et al.}\comsp}
\DeclareRobustCommand\bs{\ensuremath{\backslash}}
\newcommand\ssp{\ifmmode\relax\else\comsp\fi}

%\newcommand\Eq[1]{Eq.~(\ref{#1})}
\newcommand\Eq[1]{(\ref{#1})}
\newcommand\Fig[1]{Fig.~\ref{#1}}
\DeclareRobustCommand\NF{\ensuremath{N_{\rm f}}\ssp}
% \DeclareRobustCommand\NC{\ensuremath{N_{\rm c}}\ssp}
\DeclareRobustCommand\GeV{\ensuremath{{\rm GeV}}\ssp}
% \DeclareRobustCommand\Vstat{\ensuremath{V^{\rm (stat)}}\ssp}
% \DeclareRobustCommand\Vsys{\ensuremath{V^{\rm (sys)}}\ssp}
\DeclareRobustCommand\FL{\ensuremath{F_{\mathrm{L}}}\ssp}
\DeclareRobustCommand\FT{\ensuremath{F_{\mathrm{T}}}\ssp}
\newcommand\AP {{\cal P}}

\def \beq{\begin{equation}}
\def \eeq{\end{equation}}
\def \beqa{\begin{eqnarray}}
\def \eeqa{\end{eqnarray}}
\def \beqal{\begin{subequations}\begin{eqnarray}}
\def \eeqal{\end{eqnarray}\end{subequations}}

\let\optspace=\ssp
% \newcommand{\xh}{\hat x}
\DeclareRobustCommand{\as}[1]{\ensuremath{\alpha_{\rm s}(#1^2)}\optspace}
\newcommand{\asotp}{\ensuremath{\frac{\alpha_{\rm s}}{2\pi}}\optspace}
\newcommand{\Sgl}[1]{\ensuremath{\tilde f_{#1+}}\optspace}
\newcommand{\Pom}{{I\!P}}
\newcommand{\Reg}{{I\!R}}
% \newcommand{\xP}{x_\Pom}
\newcommand{\xP}{\xi}
\newcommand\sigRed{\ensuremath{\overline\sigma}}
\newcommand\DX{\ensuremath{\mathcal{X}}}

%\parindent=0pt
%\parskip=4pt

%\graphicspath{{figs/}}

%==========================================
\subsubsection {Cross-section}

\beq
  \frac{d\sigma}{d\beta\,dQ^2\,d\xP\,dt}
=
  \frac{2\pi\alpha^2}{\beta Q^4}\,
    \left( 1 +  (1-y)^2 \right) \sigRed^{D(4)}(\beta,Q^2,\xP,t)
\label{Dxs}
\eeq
where the `reduced cross-section', \sigRed, is defined as
\beq
\label{eq:sigred}
\sigRed
 = F_2 - \frac{y^2}{1 +  (1-y)^2}\, \FL
 = \FT + \frac{2(1-y)}{1 +  (1-y)^2}\, \FL
\eeq
Nb. $\xi$ is denoted by $x_\Pom$ in the H1 and ZEUS papers.

The dimension of 
\(
F_k^{D(4)}(\beta,Q^2,\xP,t)\)
is $\GeV^{-2}$
and
thus the quantities integrated over $t$
\beq
F_k^{D(3)}(\beta,Q^2,\xP)
\equiv
\int_{t_{\rm min}}^{t_{\rm max}} dt
F_k^{D(4)}(\beta,Q^2,\xP,t)
\eeq
are dimensionless.

Maximum kinematically allowed value of $t$ reads
\begin{equation}
t_{\rm MAX} 
=
-\frac{\xP^2 m_p^2 + p_\perp^2}{1-\xP}
\approx 
-\frac{\xP^2}{1-\xP} m_p^2
\end{equation}
where $m_p$ is the proton mass.

As $x = \xP\beta$ we can normalize to the standard DIS formula
\begin{equation}
\frac{d\sigma}{d\beta\,dQ^2\,d\xP\,dt} =
  \frac{2\pi\alpha^2}{x\, Q^4}\,
    \left( 1 +  (1-y)^2 \right) \xP\sigRed^{D(4)}(\beta,Q^2,\xP,t)
\end{equation}
which upon integration over $t$ reads
\begin{equation}
\label{Dxs3}
  \frac{d\sigma}{d\beta\,dQ^2\,d\xP}
=  
  \frac{2\pi\alpha^2}{x Q^4}\,
    \left( 1 +  (1-y)^2 \right) \,\xi\sigRed^{D(3)}(\beta,Q^2,\xP)
\end{equation}


The H1 and ZEUS data files typically contain $\xP\sigRed^{D(3)}$.

%==========================================
\subsubsection {Regge factorization}

For better data description we include a contribution from a secondary Reggeon, $\Reg$,
\beq
F_k^{D(4)}(\beta,Q^2,\xP,t) = 
\sum_{\mathcal{X} =\Pom,\Reg}
\phi_\mathcal{X}(\xP,t)\, F^\mathcal{X}_k(\beta,Q^2)
\eeq

or
\beq
\label{eq:FD3}
F_k^{D(3)}(\beta,Q^2,\xP) = 
\sum_{\mathcal{X} =\Pom,\Reg}
\Phi_\mathcal{X}(\xP)\, F^\mathcal{X}_k(\beta,Q^2)
\eeq
where
\begin{equation}
\label{eq:intFlux}
\Phi_{\mathcal{X}}(\xP) =
\int\limits_{t_{\rm min}}^{t_{\rm max}} dt\, \phi_\mathcal{X}(\xP,t)
\,.
\end{equation}

Parametrization of the fluxes
\begin{subequations}
\label{eq:flux}
\begin{equation}
\phi_\mathcal{X}(\xP,t) = 
\frac {A_\mathcal{X}\, e^{b_\mathcal{X} t}} {\xP^{2\alpha_\mathcal{X}(t) -1}}
\end{equation}
where
\begin{equation}
\alpha_\mathcal{X}(t) = \alpha_\mathcal{X}(0) + \alpha_\mathcal{X}' t
\,.
\end{equation}
\end{subequations}

$F^\Reg_k(\beta,Q^2)$ are taken as those of the pion.
%  with the normalization factor being absorbed in $\phi_\Reg(\xP,t)$.


%==========================================
\subsubsection {Pomeron parametrization}

The Pomeron is parametrized at the initial
$Q_0^2$ in terms of two singlet distributions,
$f_{g}$ and $f_{+}$.
\begin{subequations}
\label{singlet}
\begin{eqnarray}
\frac{d}{dt}f_{+} &=&
\asotp\left[
\AP_{\rm FF} f_{+} +\AP_{\rm FG}f_g
\right]
\\
\frac{d}{dt}f_{g} &=&
\asotp\left[
\AP_{\rm GF} f_{+} +\AP_{\rm GG}f_g
\right]
\end{eqnarray}
\end{subequations}

As $\Pom$ is neutral, $f_{q} = f_{\bar q}$ for each flavour $q$.
Assuming that all light quark PDFs are equal
\begin{equation}
f_d = f_u = f_s
\,,
\end{equation}
we have
\begin{subequations}
\label{eq:pm}
\begin{eqnarray}
f_{q-} &\equiv& 0
\\
f_{q+} &\equiv& 2 f_q
\end{eqnarray}
\end{subequations}

At \NF = 3
\begin{equation}
\label{eq:fq3}
f_{q+} = f_{+}/3,\; q = d,u,s
\,.
\end{equation}
\ifFullVer
\ie
\begin{equation}
\Sgl q = 0,\; q = d,u,s
\,,
\end{equation}
where
\begin{equation}
\Sgl q \equiv f_{q+} - \frac{1}{\NF}f_{+}
\,.
\end{equation}
\fi

This gives all PDFs for the FFNS, while for VFNS 
$f_{h+}, h=c,b,t$ are generated dynamically above the respective
transition scales $Q_h^2$.
Hence at \NF > 3 the singlet has contributions from the heavy quarks
and we get non-trivial nonsinglet distributions $\Sgl{h}$ satisfying
\begin{equation}
\label{eq:nsevol}
\frac{d}{dt}\Sgl h = \asotp\, \AP_{(+)} \Sgl h
\end{equation}

%\subsection {Parametrization at \texorpdfstring{$Q_0^2$}{Q0}}
%\subsubsection {Parametrization at {$Q_0^2$}}
{\bf Parametrization at {$Q_0^2$}} \\
%-----------------------------------------------------------
\label{sec:Par}

Full PDFs are given in analogy to \Eq{eq:FD3}
\begin{equation}
f_k^{D(3)}(\beta,Q^2,\xP) =
\hat\Phi_\Pom(\xP)\, f^\Pom_k(\beta,Q^2)
+
\Phi_\Reg(\xP)\, f^\Reg_k(\beta,Q^2)
\end{equation}
where $\hat\Phi_\Pom \equiv \Phi_\Pom/A_\Pom$,
with the fluxes given by \Eq{eq:intFlux} and \Eq{eq:flux}.

The Pomeron PDFs (omitting superscript $\Pom$) are parametrized as
\def\Cini#1#2{A^{(#1)}_#2}
\begin{equation}
\label{eq:fP0}
f_N = \Cini N1  x^{\Cini N2} (1-x)^{\Cini N3}
%   \left(1 + \Cini N4 x \right)
  \; \exp\left(-\frac{d}{1.00001-x}\right)
\,,
\end{equation}
where the `dumping factor' $d$ is taken as 0.01 or 0.001.
$N =$ G for gluon or S for `singlet': $f_{\rm S} \equiv f_+(\NF=3)$,
\cf \Eq{eq:fq3}.

% The Reggeon PDFs $f^\Reg_k$ are taken from pion.

%\subsubsection {HERAFitter parameters}
{\bf HERAFitter parameters} \\
%-----------------------------------------------------------
\label{sec:HFitterPar}

% minuit.in.txt
% ExtraMinimisationParameters

\begin{tabular}{l|l|l}
Parameter & HERAFitter name & input file\\
% \hline
$\Cini {\mathrm G}1$ & Ag & minuit.in.txt \\
$\Cini {\mathrm G}2$ & Bg & minuit.in.txt \\
$\Cini {\mathrm G}3$ & Cg & minuit.in.txt \\
$\Cini {\mathrm S}1$ & Auv & minuit.in.txt \\
$\Cini {\mathrm S}2$ & Buv & minuit.in.txt \\
$\Cini {\mathrm S}3$ & Cuv & minuit.in.txt \\
$\alpha_\Pom(0)$ & Pomeron\_a0 & steering.txt \\
$A_\Reg$ & Reggeon\_factor & steering.txt \\
$\alpha_\Reg(0)$ & Reggeon\_a0 & steering.txt \\
\end{tabular}




        

%%%%%%%%%%%
\section{Bayesian Reweighting Technique for the Determination of updated PDF sets}
%%%%
Bayesian reweighting of PDF sets is a way to include new data into an existing PDF set without actual carrying out a full-blown fitting procedure. It has been first suggested by Giele and Keller~\cite{Giele:1998gw} and first pursued in practice by the NNPDF Collaboration~\cite{Ball:2011gg,Ball:2010gb}. Watt and Thorne~\cite{Watt:2012tq} proposed a scheme of how to implement the Bayesian reweighting technique also for PDF predictions based on central values with errors determined using the Hessian Eigenvector Method. 

The \fitter package allows to update any PDF that is either available as probability distribution (i.e. a lhapdf .LHgrid file in NNPDF format) or as PDF eigenvector set (i.e. any PDF set in lhapdf .LHgrid file format with errors determined using the Hessian Eigenvector Method). This enables the user to assess the impact of new data not only for the {\tt HERAPDF} using the full-blown fit procedure but also for the other standard global PDF sets and allows to compare how the data impacts different PDFs.

The Bayesian Reweighting technique essentially uses PDF probability distributions as input, applies weights to these distributions based on how well the new data is described and outputs an updated PDF probability distribution. In the following paragraphs, firstly the construction of these PDF probability distributions is described, then the calculation of the weights to update the PDF probability distribution is introduced and lastly, the configuration of the module within the \fitter framework is explained.

\subsubsection{PDF probability distributions}

PDF probability distributions are constructed as finite ensembles of $N_{\mathrm{rep}}$ parton distribution functions $\mathrm{PDF}_k$, $\mathcal{E} = \{PDF_k, k = 1, . . . ,N_{\mathrm{rep}}\}$. Observables $\mathcal{O}(\mathrm{PDF})$ are conventionally calculated from the average of the predictions obtained from the ensemble:

\begin{equation}
 \langle\mathcal{O}(\mathrm{PDF})\rangle = \frac{1}{N_{\mathrm{rep}}} \sum_{k=1}^{N_{\mathrm{rep}}} \mathcal{O}(\mathrm{PDF}_k)
\label{eq:meanReplicas}
\end{equation}
 
Their uncertainties are calculated as the standard deviation, defined as:

\begin{equation}
\sigma_{\mathcal{O}(\mathrm{PDF})} = \sqrt{  \frac{1}{N_{\mathrm{rep}} - 1 }  \sum_{k=1}^{N_{\mathrm{rep}}} 
( \mathcal{O}(\mathrm{PDF}_k) - \langle \mathcal{O}(\mathrm{PDF})  \rangle   )^2     
     }
\end{equation}

While the standard PDF sets from the NNPDF collaboration are already available as ensembles of parton distribution functions, the PDF predictions of other PDF fitting groups need to be converted to PDF probability distributions. This is possible provided that the PDF sets have associated uncertainties that can be used to create replicas of the central PDF set with random variations that lie within the uncertainties. 

In the case of uncertainties provided by standard Hessian eigenvectors error sets, this can be easily achieved by creating the $k$-th random replica by introducing to the central PDF set, $\mathrm{PDF}_0$, random fluctuations. 

If the PDF eigenvectors are asymmetric, that is they come in pairs of negative and positive PDF error sets, corresponding to negative and positive deviations from the central value, these random fluctuations are created by drawing a random number $R_{jk}$ and adding, depending on the sign of the random number, the difference of the positive or respectively negative PDF of the $j$-th PDF eigenvector pair from the central value, scaled by the absolute value of the random number:

\begin{equation}
 \mathrm{PDF}_k = \mathrm{PDF}_0  + \sum_{j=0}^{n} \left[ \mathrm{PDF}^{\pm}_j - \mathrm{PDF}_0 \right] |R_{jk}|
\end{equation}
 
Here, $k$ denotes the number of the random replica and runs from $k=1, ... , N_\mathrm{rep}$; $j$ denotes the eigenvector pair and runs from $j=1, ..., n$, where $n$ is the number of eigenvectors, e.g. $n=20$ for MSTW08. 

In case, the Hessian eigenvectors are symmetrised and only one error set is given per eigenvector, the above prescription simplifies to:
   
\begin{equation}
 \mathrm{PDF}_k = \mathrm{PDF}_0  + \sum_{j=0}^{n} \left[ \mathrm{PDF}_j - \mathrm{PDF}_0 \right] R_{jk}
\end{equation}

\subsubsection{Bayesian Reweighting of PDF sets}

Once PDF probability distributions are available as inputs, they can be updated to incorporate the new data. This is achieved by applying weights to the PDF probability distributions such that the prediction for observable $\langle\mathcal{O}(\mathrm{PDF})\rangle$ from equation \ref{eq:meanReplicas} changes to:

\begin{equation}
 \langle\mathcal{O}^{\mathrm{new}}(\mathrm{PDF})\rangle = \frac{1}{N_{\mathrm{rep}}} \sum_{k=1}^{N_{\mathrm{rep}}} w_k \mathcal{O}(\mathrm{PDF}_k)
\end{equation}

The weights $w_k$ calculated are here according to:

\begin{equation}
 w_k = \frac{(\chi^2_k)^{\frac{1}{2} (N_{\mathrm{data}}-1) } \exp^{-\frac{1}{2}\chi^2_k}}{ \frac{1}{N_{\mathrm{rep}}} \sum^{N_{\mathrm{rep}}}_{k=1}(\chi^2_k)^{\frac{1}{2}(N_{\mathrm{data}}-1)} \exp^{-\frac{1}{2}\chi^2_k}  },
\end{equation}

where $N_{\mathrm{data}}$ is the number of new data points, $k$ denotes the specific replica for which the weights is calculated and $\chi^2_k$ is between a given data point $y_i$ and its theoretical prediction obtained with the $k$-th PDF replica:

\begin{equation}
 \chi^2 (y,\mathrm{PDF}_k) = \sum_{i,j=0}^{N_{\mathrm{data}}} (y_i - y_i(\mathrm{PDF}_k)) \sigma^{-1}_{ij} (y_j-y_j(\mathrm{PDF}_k))  
\end{equation}

The weighted PDF probability distribution can be turned into a new ensemble of PDF replicas, based on which predictions for any observable can be calculated. This new, reweighted PDF probability distribution commonly is chosen to be based upon a smaller number of PDF sets compared to the input PDF probability distribution, in order throw away those replicas that are incompetible with the data and to create a more light-weight PDF set. 


\subsubsection{Usage of the PDF reweighting in the \fitter framework}
 
The \fitter allows to perform PDF reweighting for NNPDF-style PDF probability distributions as well as for PDF sets with Hessian PDF error Eigenvector sets. 

This requires the {\tt NNPDF reweight} and the {\tt LHAPDF} modules to be installed, see sections \ref{sec:install_nnpdfrweight} and \ref{sec:install_lhapdf}. In the \fitter steering files, the PDF reweighting needs to be switched on and the relevant parameters have to be set: 

\begin{itemize}
 \item \textbf{FLAGRW}: En/disable reweighting
 \item \textbf{RWPDFSET}: Name of the PDF set to be reweighted
 \item \textbf{RWDATA}: Arbitrary name for the data to be updated, used to create the names of the output PDF set and the directory
 \item \textbf{RWMETHOD}: Do the reweighting based on chi2 (method 1, where you read in \fitter data files and theory predictions and calculate the chi2 based on them) or on data (method 2, where you have to provide an input text file with theoretical predictions {--} the input format of these will be explained below)
 \item \textbf{DORWONLY}: Disable the usual PDF fit, such that only the reweighting is done 
 \item \textbf{RWREPLICAS}: Number of input replicas used for the PDF probability distributions (not applicable for NNPDF sets, since they come with a fixed number of replicas)
 \item \textbf{RWOUTREPLICAS}: Number of replicas in the output PDF set.
\end{itemize}

The setup of the module is such, that it is parsing the \fitter steering file and from the specified settings creates a special reweighting steering file in the directory {\tt input\_steering} with the pattern {\tt <RWPDFSET>\_<RWDATA>\_<RWMETHOD: chi2 or data>.in}. 

In the output directory, a subdirectory is created for the output of the reweighting procedure. Its name pattern is: {\tt output/<RWPDFSET>\_<RWDATA>\_<RWMETHOD: chi2 or data>/} and it will contain the following files:

\begin{itemize}
 \item {\tt <RWPDFSET>\_<RWDATA>\_<RWMETHOD: chi2 or data>\_nRep<RWOUTREPLICAS>.LHgrid}: The output PDF probability distribution in form of an .LHgrid file, which allows easy usage.
 \item {\tt whist-rw.eps}: Plot with the distributions of weights calculated for each replica. The meaning of this variable is further described in~\cite{Ball:2011gg,Ball:2010gb}.
 \item {\tt palpha-rw.eps}: Plot with the probability for each replica to describe the data. The meaning of this variable is further described in~\cite{Ball:2011gg,Ball:2010gb}.
 \item {\tt <RWPDFSET>\_<RWREPLICAS>InputReplicas.LHgrid}: This is the PDF probability function that has been produced from the eigenvector PDF sets produced by the Hessian method (not applicable for NNPDF sets).
\end{itemize}


%%%%%%%%%%%%%%%%%%%%%%%%%%%%%%
\section{PDF Parameterisation}
\label{sec:pdfparam}
%%%%%%%%%%%
\subsection{Standard Functional form}
%%%%
\subsubsection{CTEQ style}
%%%%
\subsubsection{HERAPDF style}
%%%%
\subsubsection{Flexible style}
%%%%%%%%%%%
\subsection{Chebyshev Polynomial}

A flexible Chebyshev polynomials based parameterisation is used for the gluon and sea densities. The polynomials
use $\log x$ as an argument to emphasise the low $x$ behaviour. 
The parameterisation is valid for $x>x_{min} = 1.7\times 10^{-5}$. The PDFs are multiplied
by $1-x$ to ensure that they vanish as $x\to 1$. The resulting parameterisation form is 
\begin{eqnarray}
x g(x) &=& A_g \left(1-x\right) \sum_{i=0}^{N_g-1} A_{g_i} T_i \left(-\frac{\textstyle 2\log x - \log x_{min} } {\textstyle \log x_{min} } \right)\,, \label{eq:glu} \\
x S(x) &=& \left(1-x\right) \sum_{i=0}^{N_S-1} A_{S_i} T_i \left(-\frac{\textstyle 2\log x - \log x_{min} } {\textstyle \log x_{min} } \right)\,. \label{eq:sea} 
\end{eqnarray}
Here the sum over $i$ runs up to $N_{g,S}=15$ order Chebyshev polynomials of the first type $T_i$ for
the gluon, $g$, and sea-quark, $S$, density, respectively. 
The normalisation $A_g$ is given by the momentum sum rule.
The advantages of the parameterisation given by equations~\ref{eq:glu},\ref{eq:sea} is that momentum
sum rule can be evaluated analytically and  already for $N \ge 5$ the fit quality
is similar to a standard Regge-inspired parameterisation with a similar number of parameters.



%%%%%%%%%%%%%%%%%%%%%%%%%%%%%%
\section{$\chi^2$ Definitions}
\label{sec:chi2}
%%%%%%%%%%%
\subsection{Using Nuisance Parameters}
%%%%
\subsubsection{Simple Form}

%%%%
\subsubsection{Scaled Form}

For a single data set, 
 the $\chi^2$ function can be defined as~\cite{H1:2009bp}
%
\begin{equation}
 \chi^2_{\rm exp}\left(\boldsymbol{m},\boldsymbol{b}\right) = %\\
%~~~=
 \sum_i
 \frac{\left[m^i
- \sum_j \gamma^i_j m^i b_j  - {\mu^i} \right]^2}
{ \textstyle \delta^2_{i,{\rm stat}}\left(m^i -  \sum_j \gamma^i_j m^i b_j\right)+
\left(\delta_{i,{\rm uncor}}\,  m^i\right)^2}
 + \sum_j b^2_j.
\label{eq:ave}\end{equation}
%
Here ${\mu^i}$ is the  measured central value  at a point $i$ 
with  relative statistical $\delta_{i,stat}$ 
and relative uncorrelated systematic uncertainty $\delta_{i,unc}$.
Further, $\beta_j$ denotes a nuisance parameter for
 a correlated systematic error  source of type $j$ with an uncertainty
 while
$\gamma^i_j$ 
quantifies the sensitivity of the
measurement ${\mu^i}$ at the point $i$ to the systematic source $j$. 
The function $\chi^2_{\rm exp}$ depends on the set of
underlying physical quantities $m^i$ 
(denoted as the vector $\boldsymbol{m}$) and 
 the set of systematic uncertainties $b_j$ ($\boldsymbol{b}$).
This definition of the $\chi^2$ function takes into account that
systematic uncertainties are proportional to the central values 
(multiplicative errors), whereas the statistical errors scale 
with the square roots of the expected number of events. 
Other scaling properties for the statistical and uncorrelated
systematic uncertainties are available as described in appendix~\ref{sec:herafitter}.
%%%%
\subsubsection{Generalised Scaled Form}

%%%%%%%%%%%
\subsection{Using Covariance Matrix}
%%%%%%%%%%%%%%%%%%%%%%%%%%%%%
\section{Treatment of the Experimental Uncertainties}
%%%%%%%%%%%
\subsection{Hessian Method}

%%%%%%%%%%%
\subsection{Monte Carlo Method}


The PDF uncertainties can be estimated using a Monte Carlo technique \cite{mcmethod}.
The method consists in preparing replicas of data sets by allowing the central values of the cross sections to 
fluctuate within their systematic and statistical uncertainties taking into account all point-to-point correlations.
The preparation of the data is repeated for a large $N$ ($>100$ times) and for each of these replicas a NLO QCD fit is performed to 
extract the PDF set. The PDF central values and uncertainties are estimated using the means values and RMS 
over the replicas. 



%%%%%%%%%%%
\subsection{Regularisation methods}


%%%%%%%%%%%%%%%%%%%%%%%%%%%%%%
\section{Program Manual}
\label{sec:man}
%%%%%%%%%%%
\subsection{Steering files}
    The software behaviour is controlled by two files with steering commands.
    These files have predefined names:
    \begin{itemize}
      \item {\tt steering.txt}  --   controls main "stable" (un-modified during 
                         minimisation) parameters. The file also contains
                         names of data files to be fitted to, definition 
                         of kinematic cuts                              
      \item {\tt minuit.in.txt}
                   --  controls minimisation parameters and minimisation 
                         strategy. Standard Minuit commands can be provided
                         in this file
      \item {\tt ewparam.txt}    --  controls electroweak parameters such
         as W and Z boson masses and CKM matrix parameters.
    \end{itemize}
%%%%%%%%%%%
\subsubsection{Options for DIS fits}
%%%%
\subsubsection{Options for Jets}
%%%%
\subsubsection{Options for Diffractive fits}
%%%%
\subsubsection{Options for DY fits}
%%%%%%%%%%%
\subsection{Data file format}
\label{sec:dataformat}
   Experimental data are provided by the standard {\tt ASCII} text files. The files
   contain a "header" which describes the data format and the "data" in terms
   of a 2-dimensional table. Each line of the data table corresponds to a
   data point, the meaning of the columns is specified in the file header.

   For example, a header for HERA-I combined H1-ZEUS data for e+p neutral 
   current scattering cross section is given in the file

\begin{verbatim}
       datafiles/H1ZEUS\_NC\_e-p\_HERA1.0.dat
\end{verbatim}

   The format of the file follows standard "namelist" conventions. Comments 
   start with exclamation mark.  Pre-defined variables are:
\begin{itemize}
     \item{\tt Name}        --- (string) provides a name of the data set
    \item{\tt  Reaction}    --- (string) reaction type of the data set. Reaction type is used 
                      to trigger corresponding theory calculation. The following 
                      reaction types  are currently supported by the HERAFitter:
                      \begin{itemize}
                        \item {\tt 'NC e+-p'}  -- double differential NC ep scattering
                                      (ZMVFS and RT-VFS schemes) 
                        \item {\tt 'CC e+-p'}  -- double differential CC ep scattering
                                      (ZMVFS scheme)
                        \item {\tt 'CC pp'}    -- single differential $d \sigma_{W^{\pm}}/d eta_{\ell^{\pm}}$
                                      production and W asymmetry at $pp$ and $p\bar{p}$ 
                                      colliders (LO+kfactors and APPLGRID interface)
                        \item {\tt 'NC pp'}    -- single differential $d \sigma_Z / d y_Z$ at $pp$ and
                                      $p\bar{p}$ colliders
                                      ({\tt LO} with k-factors and {\tt APPLGRID} interface)

                        \item 'pp jets APPLGRID' -- $pp\to$ inclusive jet production, using
                                     {\tt APPLGRID}
                      \end{itemize}                       
      \item {\tt NData}       --- (integer) specifies number of data points in the file. 
                     This corresponds to the number of table rows which 
                     follow after the header.
      \item {\tt NColumn}     --- (integer) number of columns in the data table.
      \item {\tt ColumnType}  --- (array of strings)
                      Defines layout of the data table. The following column types
                      are pre-defined: 'Bin', 'Sigma', 'Error' and 'Dummy'
                      The keywords are case sensitive. 'Bin' correspond to an
                      abstract bin definition, 'Sigma' corresponds to the data
                      measurement, 'Error' - to various type of uncertainties and
                      'Dummy' indicates that the column should be ignored.
      \item {\tt ColumnName}  --- (array of strings)
                      Defines names of the columns. The meaning of the name depends
                      on the ColumnType. For ColumnType 'Bin', ColumnName gives a
                      name of the abstract bin. The abstract bins can contain
                      any variable names, but some of them must be present for 
                      correct cross section calculation. For example, 'x', 'Q2' and
                      'y' are required for DIS NC cross-section calculation.
 
                      For ColumnType 'Sigma', ColumnName provides a label for 
                      the observable, which can be any string.
 
                      For ColumnType 'Error', the following names have special meaning:
                      \begin{itemize}
                       \item 'stat'  -- specifies column with statistical uncertainties;
                       \item 'uncor' -- specifies column with uncorrelated uncertainties;  
                       \item 'total' -- specifies column with total uncertainties. 
                                  Total uncertainties are not used in the fit,
                                  however there is an additional check is performed
                                  if 'total' column is specified: sum in quadrature
                                  of statistical, uncorrelated and correlated 
                                  systematic uncertainties is compared to the total
                                  and a warning is issued if they differ significantly.
                       \item'ignore' - specifies column to be ignored (for special studies).
                       \item Other names specifies columns of correlated systematic 
                      uncertainty. For a given data file, each column of the correlated
                      uncertainty must have unique name. To specify correlation across
                      data files, same name must be used for different files.  
                      \end{itemize}
      \item {\tt SystScales}  --- (array of float)
                      For special studies, systematic uncertainties can be scaled
                      The numbering of uncertainties starts from the first column
                      with the ColumnType 'Error'. For example, setting 
\begin{verbatim}
                  SystScale(1) = 2.  
\end{verbatim}
                      in {\tt datafiles/H1ZEUS\_NC\_e-p\_HERA1.0.dat} would scale stat. 
                      uncertainty by factor of two.                       
      \item {\tt Percent}     --- (array of bool) For each uncertainty specify if it is given in 
                      absolute ("false") or in percent ("true").  The numbering of 
                      uncertainties starts from the first column with the 
                      {\tt ColumnType} 'Error' (see example above).
      \item {\tt NInfo}       --- (integer) Calculation of the cross-section predictions may 
                      require  additional information about the data set. The number of 
                      information strings is given by NInfo
      \item {\tt CInfo}       --- (array of strings) Names of the information strings. 
                      Several of them are predefined for different cross-section 
                      calculations.
      \item {\tt DataInfo}    --- (array of float) Values, corresponding to {\tt CInfo} names.
      \item {\tt IndexDataset} -- (integer) Internal H1 Fitter index of the data set. Provide unique
                      numbers to get extra info for $\chi^2/dof$ for each data set.      
      \item {\tt TheoryInfoFile} --- (string) Optional additional theory file with extra 
                     information for cross-section calculation. This could be k-factors,
                     {\tt APPLGRID} file or {\tt FastNLO} table.  
      \item {\tt TheoryType} --- (string) Theory file type ('kfactor', 'applgrid' or 'fastnlo').      
      \item {\tt NKFactor}   --- (integer) For kfactor files, number of columns in
                     {\tt TheoryInfoFile}.
      \item {\tt KFactorNames} --- (array of strings) For kfactor files, names of columns in 
                     {\tt TheoryInfoFile}.
\end{itemize}

%%%%%%%%%%%
\subsection{Selection of the data}
  The namelist \&Cuts, located inside the {\tt steering.txt} file can be used to apply
  simple process dependent cuts. The cuts are limitted to bin variables.
  Simple low and high limits are allowed. For example, a cut on $Q^2>3.5$~GeV$^2$ for
  NC ep scattering is specified as

\begin{verbatim}
  ! Rule #1: Q2 cuts
   ProcessName(1)     = 'NC e+-p'
   Variable(1)        = 'Q2'
   CutValueMin(1)     = 3.5 
   CutValueMax(1)     = 1000000.0
\end{verbatim}

  Maximum 100 cuts can be used by default.
%%%%%%%%%%%
\subsection{Understanding the output}
  The results of the minimization are printed to the standard output and written
  to the files in the {\tt output/} directory. 

  The quality of the fit can be judged based on total $chi^2$ per degrees of freedom.
  It is printed for each iteration as 
\begin{verbatim}
                      Iteration   Chi2   NDF       Chi2/NDF
   FitPDF f,ndf,f/ndf      3      588.64 579        1.02
\end{verbatim}
  The resulting $chi^2$ is reported at the end of minimisation for each data set and for correlated 
  systematic uncertainties separately. This information is printed and written
  to the {\tt output/Results.txt} file. The {\tt Results.txt} file contains additional 
  information about shifts of the correlated systematic uncertainties.

  The minimization information from the {\tt minuit} program is stored using the standard {\tt minuit} in the {\tt output/minuit.out.txt}
  file. The level of verbosity for this information can be changed by {\tt minuit} commands
  in the {\tt minuit.in.txt} file. Make sure that {\tt minuit} does not report any errors
  or warnings at the end of minimisation.
  
  Point by point comparison of the data and predictions after the minimization 
  is provided in {\tt output/fittedresults.txt} file. The file reports three columns
  corresponding to the three first bins of the input tables, data value, sum in 
  quadrature of statistical and uncorrelated systematic uncertainty, total
  uncertainty, the predicted value, before and after applying correlated systematic shifts,
  pull betweenthe  data and theory and 
  data set index. The pull $p$ is calculated as 
  \begin{equation}
      p = \frac{ \mu - m} {\sigma_{\rm uncor}}
  \end{equation}
  where $\mu$ is the data value, $m$ is the prediction and $\sigma_{\rm uncor}$ is the total
  uncorrelated uncertainty.
  Similar information is stored in the {\tt pulls.first.txt} and {\tt pulls.last.txt} files
  ( dataset index, first bin, second bin, third bin, theory, data, pull).
  Theory is  adjusted for systematic error shifts in this case.

  The output PDFs are stored in  {\tt output/pdfs\_q2val\_XX.txt} files.
  Each of the files reports values of gluon, and quark PDFs as a function of $x$
  for fixed $Q^2$ points. The $Q^2$ values and $x$ grid are specified by 
  {\tt \&Output} namelist in the {\tt steering.txt} file.
  
  The PDF information and data to theory comparisons can be plotted using 
  the {\tt bin/DrawResults} program.  Calling it without arguments plots results from
  {\tt output/} directory. Given the program one argument specifies sub-directory 
  where the information is read. Calling the {\tt bin/DrawResults} program with two
  arguments provides comparison of the PDFs obtained in the two fits.
  
  Finally, the \fitter\ package provides PDFs in the {\tt LHAPDF} format. To obtain the
  {\tt LHAPDF} grid file, run the {\tt tools/tolhapdf.cmd} script. The script produces 
  the {\tt PDFs.LHgrid} file which can be read by the lhapdf version lhapdf-5.8.6.tar.gz
  or later.
%%%%%%%%%%%
\subsection{{\tt Minuit} steering cards}

%%%%%%%%%%%%%%%%%%%%%%%%%
\section{User Examples}
%%%%
\subsection{DIS inclusive only}

%%%%
\subsection{All processes}


%%%%%%%%%%%%%%%%%%%%%%%%%
%%%%%%%%%%%%%%%%%%%%%%%%%
\bibliography{writeup.bib}
%%%%%%%%%%%%%%%%%%%%%%%%%
\appendix
%%%%%%%%%%%%%%%%%%%%%%%%%
\section{{\tt \&HERAFitter} namelist format}
\label{sec:herafitter}
\begin{itemize}
  \item {\tt ITheory} --- (integer) Currently only QCDNUM standard evolution
     is implemented for which {\tt ITheory} is set to 0.
  \item {\tt IOrder} --- (integer) For {\tt ITheory} =0 (collinear factorisation) : 
        LO fit (1) or NLO (2) or NNLO (3) 
  \item {\tt Q02} --- (float) Evolution starting scale.
  \item {\tt HF\_SCHEME} --- Specify heavy quark flavour treatment for neutral
 current $ep$ process. The following schemes are implemented: 
    \begin{itemize}
      \item {\tt 'ZMVFNS'}: Zero Mass Variable Flavour Number Scheme, as implemented
 in {\tt QCDNUM}.
      \item {\tt 'RT'}: Thorne-Roberts VFN scheme for $F_2^{\gamma}$. 
      \item {\tt 'RT FAST'}: Fast approximate RT VFN scheme using k-factor 
with respect ot QCDNUM ZMVFNS, calculated at the first iteration.
    \end{itemize}
\item {\tt PDFStyle} --- (string) PDF parameterisation style. Possible styles are currently available:
   \begin{itemize}
  \item{\tt '10p HERAPDF'} -- HERAPDF-like with an extra assumption 
                                 $B_{u_v} = B_{d_v}$;
  \item{\tt '13p HERAPDF'} -- HERAPDF-like with $B_{u_v}$ and $B_{d_v}$ 
                          floated independently;
  \item{\tt '10p H12000'}  -- H12000-like with independent PDFs being the
               $D,U,\bar{D},\bar{U}$ quarks and gluon.
  \item{\tt 'CTEQ'}        -- CTEQ-like parameterisation.
  \item{\tt 'CHEB'}        -- CHEBYSHEV parameterisation based on 
         gluon,sea, $u_{v}$, $d_{v}$ independent pdfs.
 \end{itemize}
\item {\tt CHI2Style}  --- (string) choice of the $\chi^2$ function:
   \begin{itemize}
   \item {\tt 'H12000'} -- Pascaud-like, systematic shifts to theory, no scaling of statistical, uncorrelated errors.
   \item {\tt 'HERAPDF'} -- Pascaud-like + "mixed error scaling"
   \item {\tt 'HERAPDF Sqrt'}   -- Pascaud-like + "sqrt error scaling"
   \item {\tt 'HERAPDF Linear'} -- Pascaud-like + "linear error scaling"
 \end{itemize}
  \item {\tt LDEBUG}  --- (logical) debug flag.
\end{itemize}
%%%%%%%%%%%%%%%%%%%%
\section{How do add new module}
\subsection{Structure of the module}
\subsection{Theory Interface Example}
\subsection{Steerings and Configurations}
%%%%%%%%%%%%%%%%%%%%%
\section{How to add new data}
Inclusion of the data files is controlled by {\tt \&InFiles} namelist in the 
{\tt steering.txt} file. For example, by default the following four HERA-I
    files are included:
\begin{verbatim}
&InFiles
    NInputFiles = 4
    InputFileNames(1) = 'datafiles/H1ZEUS_NC_e-p_HERA1.0.dat'
    InputFileNames(2) = 'datafiles/H1ZEUS_NC_e+p_HERA1.0.dat'
    InputFileNames(3) = 'datafiles/H1ZEUS_CC_e-p_HERA1.0.dat'
    InputFileNames(4) = 'datafiles/H1ZEUS_CC_e+p_HERA1.0.dat'
&End
\end{verbatim}

To include more files:
\begin{itemize}
 \item  Increase the {\tt NInputFiles} variable.
 \item  Specify the additional file by providing corresponding
  {\tt InputFileNames()} variable.
\end{itemize}
Details about data file format can be found in section~\ref{sec:dataformat}.



\end{document}

