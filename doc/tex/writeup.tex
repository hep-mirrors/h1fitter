\RequirePackage{lineno} 
\documentclass[11pt,twoside,a4paper]{article}
\usepackage{graphicx,epsfig}
\usepackage{hhline}

\usepackage{amsmath,amssymb}
\usepackage{times}
\usepackage[varg]{txfonts}
\DeclareMathAlphabet{\mathbold}{OML}{txr}{b}{it}

\usepackage{array,multirow,dcolumn}
%\usepackage[mathlines,displaymath]{lineno}
\usepackage{rotating}

\usepackage[hypertexnames,setpagesize,%
    pdftex,%
    colorlinks,%
    citecolor=blue,%
    hyperindex,%
    plainpages=false,%
    bookmarksopen,%
    bookmarksnumbered%
  ]{hyperref}

% --- DO NOT remove this line:
\providecommand\texorpdfstring[2]{#1}

% we use natbib instead of cite to work with hyperref
%\usepackage{cite}
\usepackage[numbers,square,comma,sort&compress]{natbib}
%\usepackage{hypernat}
\usepackage{textcomp}

\bibliographystyle{plain}
\renewcommand{\topfraction}{1.0}
\renewcommand{\bottomfraction}{1.0}
\renewcommand{\textfraction}{0.0}

% \renewcommand{\arraystretch}{1.2}
\newlength{\dinwidth}
\newlength{\dinmargin}
\setlength{\dinwidth}{21.0cm}
\textheight24cm \textwidth16.0cm
\setlength{\dinmargin}{\dinwidth}
\setlength{\unitlength}{1mm}
\addtolength{\dinmargin}{-\textwidth}
\setlength{\dinmargin}{0.5\dinmargin}
\oddsidemargin -1.0in
\addtolength{\oddsidemargin}{\dinmargin}
\setlength{\evensidemargin}{\oddsidemargin}
\setlength{\marginparwidth}{0.9\dinmargin}
\marginparsep 8pt \marginparpush 5pt
\topmargin -42pt
\headheight 12pt
\headsep 30pt \footskip 32pt
\parskip 3mm plus 2mm minus 2mm


\newcommand\fitter{ \mbox{\tt HERAFitter} }
\title{\fitter\ - PDF Fitting package}
\author{HERAFitter developers}
\begin{document}
\maketitle
\begin{abstract}
\end{abstract}
\tableofcontents
\linenumbers
\newpage
%%%%%%%%%%%%%%%%%%%%%%%%%%%%%%
\section{Introduction}
\label{section:introduction}
%%%%%%%%%%%%%%%%%%%%%%%%%%%%%%
This manual provides a short description of the \fitter\ program 
which can be used to determine unpolarised parton density functions 
(PDFs) using deep inelastic scattering (DIS) data and other processes such as 
Drell-Yan, jet or ttbar processes.
The parton density functions are needed to calculate cross sections
for the $ep$ and $pp$ colliders and thus required for interpetation
of the data collected at the LHC.
% The \fitter\ program were used to determine the HERA1.0 PDF set~\cite{h1zeus:2009wt}.

A schematic structure of the \fitter\ is illustrated in Fig.~\ref{fig:flow}
\begin{figure}
\begin{center}
\caption{Schematic Structure of the \fitter\ program}
\includegraphics[width=0.75\linewidth]{figures/flow.pdf}
\end{center}
\label{fig:flow}
\end{figure}

The manual is structured such that it first dedcribes briefly the theoretical input and fit choices
and then it provides the user manual with concrete examples.
Therefore, the manual begins with program installation instructions for different scenarios (section~\ref{sec:install}), followed by a brief discussion of the theoretical calculation
used in the program (section~\ref{sec:theory}) continued by a description of the
PDF parameterisation (section~\ref{sec:pdfparam}) and various $\chi^2$ functions used in the
minimisation (section~\ref{sec:chi2}). A description of the program steering cards and
the output options is given in section~\ref{sec:man}.

  
%%%%%%%%%%%%%%%%%%%%%%%%%%%%%%
\section{Theoretical Input}
\input{Theory}

%%%%%%%%%%%%%%%%%%%%%%%%%%%%%
\section{Bayesian Reweighting Technique}
\input{Reweighing} 
%%%%%%%%%%%%%%%%%%%%%%%%%%%%%
\section{PDF Parameterisation}
\input{PDFparametrisation}

\section{Chisquare  Definition}
\input{Chisquare}

\section{Treatment of the Experimental Uncertainties}
\input{ErrorTreatment}
%%%%%%%%%%%%%%%%%%%%%%%%%%%%%%

\newpage
\section{Program Installation Instructions} 
\input{Installation}

\section{Program Manual}
\input{ProgramManual}
%%%%%%%%%%%%%%%%%%%%%%%%%%%%%%
\section{User Example}
\input{UsersExample}
%%%%%%%%%%%%%%%%%%%%%%%%%%%%%%

\bibliography{writeup.bib}
%%%%%%%%%%%%%%%%%%%%%%%%%%%%%%

\appendix
%%%%%%%%%%%%%%%%%%%%%%%%%%%%%%
\section{\fitter\ Namelist}
\input{Namelist}

\section{How to add new data}
Inclusion of the data files is controlled by {\tt \&InFiles} namelist in the 
{\tt steering.txt} file. For example, by default the following four HERA-I
    files are included:
\begin{verbatim}
&InFiles
    NInputFiles = 4
    InputFileNames(1) = 'datafiles/H1ZEUS_NC_e-p_HERA1.0.dat'
    InputFileNames(2) = 'datafiles/H1ZEUS_NC_e+p_HERA1.0.dat'
    InputFileNames(3) = 'datafiles/H1ZEUS_CC_e-p_HERA1.0.dat'
    InputFileNames(4) = 'datafiles/H1ZEUS_CC_e+p_HERA1.0.dat'
&End
\end{verbatim}

To include more files:
\begin{itemize}
 \item  Increase the {\tt NInputFiles} variable.
 \item  Specify the additional file by providing corresponding
  {\tt InputFileNames()} variable.
\end{itemize}
Details about data file format can be found in section~\ref{sec:dataformat}.
%%%%%%%%%%%%%%%%%%%%
\section{How do add new module}
\subsection{Structure of the module}
\subsection{Theory Interface Example}
\subsection{Steerings and Configurations}
%%%%%%%%%%%%%%%%%%%%%




\end{document}

