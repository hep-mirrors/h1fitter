\label{sec:chi2}
%%%%%%%%%%%
\subsection{Using Nuisance Parameters}
%%%%
\subsubsection{Simple Form}
\begin{equation}
 \chi^2_{\rm exp}\left(\boldsymbol{m},\boldsymbol{b}\right) = %\\
%~~~=
 \sum_i
 \frac{\left[m^i
- \sum_j \gamma^i_j m^i b_j  - {\mu^i} \right]^2}
{ \textstyle \left(\delta_{i,{\rm stat}}m^i\right)^2 +
\left(\delta_{i,{\rm uncor}}\,  m^i\right)^2}
 + \sum_j b^2_j.
\label{eq:ave}\end{equation}
%

%%%%
\subsubsection{Scaled Form}

For a single data set, 
 the $\chi^2$ function can be defined as~\cite{H1:2009bp}
%
\begin{equation}
 \chi^2_{\rm exp}\left(\boldsymbol{m},\boldsymbol{b}\right) = %\\
%~~~=
 \sum_i
 \frac{\left[m^i
- \sum_j \gamma^i_j m^i b_j  - {\mu^i} \right]^2}
{ \textstyle \delta^2_{i,{\rm stat}}\mu^i \left(m^i -  \sum_j \gamma^i_j m^i b_j\right)+
\left(\delta_{i,{\rm uncor}}\,  m^i\right)^2}
 + \sum_j b^2_j.
\label{eq:ave}\end{equation}
%
Here ${\mu^i}$ is the  measured central value  at a point $i$ 
with  relative statistical $\delta_{i,stat}$ 
and relative uncorrelated systematic uncertainty $\delta_{i,unc}$.
Further, $\beta_j$ denotes a nuisance parameter for
 a correlated systematic error  source of type $j$ with an uncertainty
 while
$\gamma^i_j$ 
quantifies the sensitivity of the
measurement ${\mu^i}$ at the point $i$ to the systematic source $j$. 
The function $\chi^2_{\rm exp}$ depends on the set of
underlying physical quantities $m^i$ 
(denoted as the vector $\boldsymbol{m}$) and 
 the set of systematic uncertainties $b_j$ ($\boldsymbol{b}$).
This definition of the $\chi^2$ function takes into account that
systematic uncertainties are proportional to the central values 
(multiplicative errors), whereas the statistical errors scale 
with the square roots of the expected number of events. 
Other scaling properties for the statistical and uncorrelated
systematic uncertainties 
are discussed later.



In the case of off-diagonal statistical uncertainties, the $\chi^2$ function
is
\begin{equation} \label{eq:chi2gen}
\chi^2_{\rm exp} (\boldsymbol{m},\boldsymbol{b}) = \sum_{ij} \left ( m^i - \sum_l \Gamma^i_l(m^i)b_l - \mu^i \right)
  C^{-1}_{{\rm stat.}~ij}(m^i,m^j) \left(  m^j - \sum_l \Gamma^j_l(m^j)b_l - \mu^j \right) + 
\sum_l b^2_l \,.
\end{equation}
Here the scaling properties of the correlated systematic uncertainties 
$\Gamma^i_j$ and
of the covariance matrix $C_{{\rm stat.}~ij}$ are expresses as a dependence
on $m_i$ and the dependence of $\Delta_{\rm stat}$ on $b_j$ is ignored.

Eq.~\ref{eq:chi2gen} allows for two methods for fast determination
of the minimum, without need to include the formal nuisance parameters
corresponding to the systematic error sources into the minuit minimisation.
In the first method, the minimisation vs. $b_j$ is used to define covariance
matrix for the systematic uncertainties which is determined as
\begin{equation}
 C_{{\rm syst}~ij}= \sum_l \Gamma^i_l \Gamma^j_l \,.
\end{equation}
The total covariance matrix is given by the sum of the statistical and
systamtic covariance matrices
\begin{equation} 
C_{{\rm tot}~ij} = C_{{\rm stat}~ij} + C_{{\rm syst}~ij}\,,
\end{equation}
and the $\chi^2$ function takes a form
\begin{equation}
  \chi^2( \boldsymbol{m}) = \sum_{ij} ( m^i - \mu^i) C^{-1}_{{\rm tot}~ij} 
( m^j - \mu^j)\,.
\end{equation}

The second methods is used to determine optimal shifts of the nuisance
parameters at each iteration. The shifts are given by minimising 
Eq.~\ref{eq:chi2gen} vs. $b_l$ which leads to a system of  linear equations 
\begin{equation}
 \sum_k \sum_{ij} C^{-1}_{{\rm stat}~ij} \Gamma^i_l \Gamma^j_k \cdot b_k = \sum_{ij} C^{-1}_{{\rm stat}~ij} \Gamma^i_l (m^j - \mu^j)\,,
\end{equation}
where $1\le l \le N_{\rm syst}$, the total number of correlated systematic uncertainties.

Finally the nuisance parameters $\boldsymbol{b}$ can be excluded from the $\chi^2$ minimisation.  
In this case, which is referred to as an Offset method, the minimum is determined for their values set to zero
while uncertainties on the parameters $\boldsymbol{p}$ are determined by shifting each nuisance parameter $b_l$
by $\pm 1$. The total covariance matrix for parameters $p^i$ is determined as 
\begin{equation}
  C^{\rm offset}_{ {\rm par}~ ij} = \sum_{l=1}^{N_{syst}} \Delta p^i_l \Delta p^j_l \,,
\end{equation}
where $ \Delta p^i_l = 0.5 ( p^i( b_l = +1 ) - p^i(b_l = -1))$ and the quality of the fit is estimated by 
fixing $\boldsymbol{p}$ to the value at the minimum and minimising with respect to $\boldsymbol{b}$

Finally, all three approaches can be combined together. For example, only some of the systematic uncertainties
can be treated using the matrix method while others can be treated using the hessian method. In this case, the
covariance matrix  $C_{\rm syst}$ is build using the corresponding sub-set of systematic sources and $C_{\rm stat}$ 
is replaced by $C_{\rm stat}+C_{\rm syst}$ in Eq.~\ref{eq:chi2gen}. Similarly, some of the systematic uncertainties
can be treated using offset method and then $C^{\rm total}_{ {\rm par}} = C^{\rm hessian}_{\rm par} + C^{\rm offset}_{\rm par}$
where offset and hessian covariance matrices are calculated using corresponding systematic error sources.

\subsection{Bias corrections}

The correlated and uncorrelated systematic uncertainties can be treated as additive,  $\Gamma^i_l(m^i) = \gamma^i_l \mu^i$
or multiplicative, $\Gamma^i_l(m^i) = \gamma^i_l m^i$. The LogNormal treatment in which 
$ \mu^i + \sum_l \Gamma^i_j b_l$ is replaced by $ \mu^i \prod_l \exp( \gamma^i_j b_l) $ is forseen for the
next release of the {\tt HERAFitter}. 

The statistical uncertainties can be treated as additive, $\Delta^i(m^i) = \delta^i \mu^i$  and as Poisson,
$\Delta^i(m^i) = \delta^i \sqrt{\mu^i m^i}$. More complex scaling from Eq.~\ref{eq:ave}, 
which depends on shifts of $b_j$, is implemented using an iterative approach: for the first iteration $b_l =0$ 
 is used to determine values of $b_l$ which are then applied in the second iteration. Statistical covariance
matrix is scaled in a similar manner. In this case the correlation matrix is assumed to be fixed, the diagonal
ellements are updated using the prescription describe above and the covariance matrix is rescaled accordingly.

The modifications of the covariance matrix at each iteration of the minuit minimisation may lead to systematic
biases. There are two approaches to avoid these biases. In the first approach the covariance matrix is calculated
using the expected values at the first iteration of the minimisation and kept fixed to these values for further
iterations. This method requires several repetitions of the minimisation, to ensure that values close to optimal
are obtained already at the first iteration. The second method modifies the $\chi^2$ function by adding a term
corresponding to non-constant value of the covariance matrix:
\begin{equation}
 \chi^2_{\rm log} = 2 \log \frac{\Delta^i(m^i)}{\Delta^i(\mu^i)} 
\end{equation}  

\subsection{HERAFitter implementation}
 
% available as described in appendix~\ref{sec:herafitter}.
%%%%
%\subsubsection{Generalised Scaled Form}

%%%%%%%%%%%
%\subsection{Using Covariance Matrix}
%%%%%%%%%%%%%%%%%%%%%%%%%%%%%
