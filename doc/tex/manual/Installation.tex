
\label{sec:install}
%%%%%%%%%%%

To install {\fitter} together with the most commonly used modules it is suggested to use the default installation script:

{\scriptsize
{\tt https://wiki-zeuthen.desy.de/xFitter/xFitter/DownloadPage?action=AttachFile\&do=view\&target=install-xfitter}
}

If you want to install the package manually, use the following instruciton.

\subsubsection{Core Installation and Modules}
The Installation Instructions are dependent on which modules are activated via the configuration option. To see complete list of options and possible
modules, run
\begin{verbatim}
./configure --help
\end{verbatim}
\subsubsection{Pre-requirements}

The \fitter\ program has been tested on various platforms:\\
 SL6 (64 bit),  Ubuntu 14.04, Mac OS.  The following programs and system libraries are required:
\begin{itemize}
 \item {\tt gfortran}, {\tt gcc} and {\tt g++}  comiliers.   
 \item {\tt lapack}, {\tt blas} libraries
\end{itemize}

The following package is required in order to build the \fitter\ package:
\begin{itemize}
\item QCDNUM~\cite{qcdnum} versions starting from {\tt qcdnum-17-01-11} should be used. These can be found at \\
  {\tt http://www.nikhef.nl/$\sim$h24/qcdnum/QcdnumDownload.html}
\end{itemize}

%%%%%%%%%%%
\subsubsection{Default Minimal Installation}
\begin{itemize}
\item
 Makesure that \qcdnum installation directory is added to the executabel
 path and that {\tt LD\_LIBRARY\_PATH} points to the \qcdnum libraries.
 To verify that, type
\begin{verbatim}
 qcdnum-config  --libdir
\end{verbatim}
and check that the printed directory is indeed included in the {\tt LD\_LIBRARY\_PATH} list.

\item Run:
\begin{verbatim}
    ./configure
    make 
    make install
\end{verbatim}
If your system does not have {\tt root} analysis package installed, you can use the {\fitter} core functionality, however certain packages such as {\tt xfitter-draw} 
can not be built. The support of these programs is included by default and thus 
{\tt root} package is required when you run {\tt configure} script without 
additional options. You may disable root by configuring using 
{\tt configure --disable-root} option.


After these commands are finished, the executable {\tt bin/xfitter} 
file should be installed
\item  Run a check:
\begin{verbatim}
    bin/xfitter 
\end{verbatim}
\end{itemize}
%%%%%%%%%%%
\subsubsection{Installation with external packages
( \applgrid, \apfel, \mela, \lhapdf )}
\label{sec:installation}
\begin{itemize}
\item Make sure that \qcdnum bin directory and libraries are included
in the paths as described in the previous section
\item Make sure that {\tt \$PATH} and {\tt \$LD\_LIBRARY\_PATH} 
variables point to the external package(s) enviroment.
\item Run:
\begin{verbatim}
   # For applgrid only:
    ./configure --enable-applgrid 
   # For mela only:  
   # ./configure --enable-mela 
   # For apfel only:
   # ./configure --enable-apfel
   # For lhapdf only 
   # ./configure --enable-lhapdf
   # For all packages:
   #  ./configure --enable-applgrid  --enable-mela  --enable-apfel  --enable-lhapdf

    make 
    make install
\end{verbatim}
After these commands are finished, the executable {\tt bin/xfitter} 
file should be installed
\item  Run a check:
\begin{verbatim}
    bin/xfitter 
\end{verbatim}
\end{itemize}
%%%%%%%%%%%


%%%%%%%%%%%
\subsubsection{Installation with {\tt HATHOR}}
 Note that support of {\tt HATHOR} is suspended in {\fitter}. Please use
 this option with extra caution.
 \begin{itemize}
  \item Download Hathor from 
\begin{verbatim}
http://www-zeuthen.desy.de/~moch/hathor/
\end{verbatim}
     and install it according to the instructions given there
     (requires \verb'LHAPDF' library)

  \item Define a variable HATHOR\_ROOT  such that HATHOR\_ROOT  points to the
     directory of your Hathor installation

  \item Install the \fitter\ as described above but configuring it
     with the option "--enable-hathor" before building it
 \end{itemize}


%%%%%%%%%%%
%\subsection{Installation with {\tt CASCADE}}
\subsubsection{Installation for TMD (uPDF) in high-energy factorisation (using  {\tt CASCADE})}

\begin{itemize}

\item Installation with TMD requires Cascade and Pythia generators, 
they can be downloaded from\\
{\tt http://cascade.hepforge.org/ } and 
{\tt https://pythia6.hepforge.org/ } respectively. \\
        \\
After installation of the generator packages, the {\tt CASCADE\_ROOT}  and {\tt PYTHIA\_ROOT} 
environment variables have be specified and point to the corresponding libraries. 
In the DESY afs environment the pre-installed versions of Cascade and Pythia can be used:  
%
{\footnotesize\begin{verbatim}
export CASCADE\_ROOT=/afs/desy.de/group/alliance/mcg/public/MCGenerators/cascade/2.2.04/\$SYSNAME 
export PYTHIA\_ROOT=/afs/desy.de/group/alliance/mcg/public/MCGenerators/pythia6/425/\$SYSNAME}
\end{verbatim} }
\normalsize
where {\tt SYSNAME } = i586\_rhel50 or similar.

\item Run:
\begin{verbatim}
    ./configure --enable-updf --enable-lhapdf
    make 
    make install
\end{verbatim}


\item use steering and \minuit input files from "input\_steering": 

   \begin{verbatim} 
   cp input-steering/steering.txt.kt-factorisation steering.txt 
   cp input-steering/minuit.in.txt.kt-factorisation minuit.in.txt 
   cp input-steering/steer-ep-CASCADE steer-ep 
   cp input-steering/steer_gluon-evolv steer_gluon-evolv
    \end{verbatim}

\item  edit steering.txt: 
   \begin{verbatim}
   \&CCFMFiles: give name for output grid file for uPDF   
   \&\fitter\ 
   TheoryType = 'uPDF3' ! 'DGLAP'  -- collinear evolution
                ! 'uPDF'   -- un-integrated PDFs:
                !  uPDF1 fit with kernel ccfm-grid.dat file
                !  uPDF2 fit evolved uPDF, fit just normalisatio
                !  uPDF3 fit using precalculated grid of sigma_hat
                !  uPDF4 fit calculating kernel on fly, grid of sigma_hat
  \end{verbatim}
  The recommended option is \verb+uPDF4+, which evolves the evolution kernel for gluons and valence quarks
  (evolution parameters are set in \verb+steer_gluon-evolv+). After evolution of the kernel, $\hat {\sigma}$ is
  calculated in a grid in $x$ at the  $Q^2$ values used in the data sets selected. The ${\hat \sigma}$ values are   
  stored for transverse and longitudinal cross sections for light quarks ($n_f \leq 3$), charm and beauty quarks.
\item run the program: bin/xfitter 
   
\item plotting $F_2$ fit results: \\
xfitter-draw  can be used to draw $F_2$ results. \\
The uPDFs need to be plotted with an external package (currently not available).
\end{itemize}

