In this section a alternative approaches to PDF studies are described. These are PDF profiling and reweighting. The later allows to update probability distribution based PDF (e.g. NNPDF PDF sets) to be updated with new data inputs. PDF profiling is a bit more universal, it compares data and MC predictions and based on the $\chi^2$-minimization technique implemented in the \fitter, it constraints the individual PDF eigenvector sets of the input PDFs taking into account also the data uncertainties. Both of these techniques are implemented in the \fitter and are described below.


\subsection{Bayesian Reweighting}

Bayesian reweighting of PDF sets is a way to include new data into an existing PDF set without actually carrying out a full-blown fitting procedure. 
It was first suggested by Giele and Keller~\cite{Giele:1998gw} and first pursued in practice by the NNPDF Collaboration~\cite{Ball:2011gg,Ball:2010gb}. 
Watt and Thorne~\cite{Watt:2012tq} have also proposed a scheme for how to implement the Bayesian reweighting technique for PDF predictions based on central values with errors determined using the Hessian Eigenvector Method. 

allows these methods to be used to update any PDF that is available either as a probability distribution... or as a PDf eigenvector set

The \fitter package allows these methods to be used to update any PDF that is available either as a probability distribution
(i.e. a lhapdf .LHgrid file in NNPDF format) or as a PDf eigenvector set 
(i.e. any PDF set in lhapdf .LHgrid file format with errors determined using the Hessian Eigenvector Method).
This enables the user to assess the impact of new data not only on the {\tt HERAPDF} using the full-blown fit procedure 
but also on other standard PDF sets. This one can investigate how the data impact different PDF sets.

The Bayesian Reweighting technique essentially uses PDF probability distributions as input, applies weights to these distributions based on how well the new data is described and outputs an updated PDF probability distribution. In the following paragraphs, firstly the construction of these PDF probability distributions is described, then the calculation of the weights to update the PDF probability distribution is introduced and lastly, the configuration of the module within the \fitter framework is explained.

\subsubsection{PDF probability distributions}

PDF probability distributions are constructed as finite ensembles of $N_{\mathrm{rep}}$ parton distribution functions $\mathrm{PDF}_k$, $\mathcal{E} = \{PDF_k, k = 1, . . . ,N_{\mathrm{rep}}\}$. Observables $\mathcal{O}(\mathrm{PDF})$ are conventionally calculated from the average of the predictions obtained from the ensemble:

\begin{equation}
 \langle\mathcal{O}(\mathrm{PDF})\rangle = \frac{1}{N_{\mathrm{rep}}} \sum_{k=1}^{N_{\mathrm{rep}}} \mathcal{O}(\mathrm{PDF}_k)
\label{eq:meanReplicas}
\end{equation}
 
Their uncertainties are calculated as the standard deviation, defined as:

\begin{equation}
\sigma_{\mathcal{O}(\mathrm{PDF})} = \sqrt{  \frac{1}{N_{\mathrm{rep}} - 1 }  \sum_{k=1}^{N_{\mathrm{rep}}} 
( \mathcal{O}(\mathrm{PDF}_k) - \langle \mathcal{O}(\mathrm{PDF})  \rangle   )^2     
     }
\end{equation}

While the standard PDF sets from the NNPDF collaboration are already available as ensembles of parton distribution functions, the PDF predictions of other PDF fitting groups need to be converted to PDF probability distributions. This is possible provided that the PDF sets have associated uncertainties that can be used to create replicas of the central PDF set with random variations that lie within the uncertainties. 

In the case of uncertainties provided by standard Hessian eigenvector error sets, this can be easily achieved 
by creating the $k$-th random replica by introducing introducing random fluctuations around the central PDf set, $\mathrm{PDF}_0$.

If the PDF eigenvectors are asymmetric, that is they come in pairs of negative and positive PDF error sets, 
corresponding to negative and positive deviations from the central value, these random fluctuations can be created by 
drawing a random number $R_{jk}$ and adding, depending on the sign of the random number, 
the difference of the positive or respectively negative PDF of the $j$-th PDF eigenvector pair from the central value, 
scaled by the absolute value of the random number:

\begin{equation}
 \mathrm{PDF}_k = \mathrm{PDF}_0  + \sum_{j=0}^{n} \left[ \mathrm{PDF}^{\pm}_j - \mathrm{PDF}_0 \right] |R_{jk}|
\end{equation}
 
Here, $k$ denotes the number of the random replica and runs from $k=1, ... , N_\mathrm{rep}$; $j$ denotes the eigenvector pair and runs from $j=1, ..., n$, where $n$ is the number of eigenvectors, e.g. $n=20$ for MSTW08. 

In case that the Hessian eigenvectors have been symmetrised and only one error set is given per eigenvector, 
the above prescription simplifies to:

\begin{equation}
 \mathrm{PDF}_k = \mathrm{PDF}_0  + \sum_{j=0}^{n} \left[ \mathrm{PDF}_j - \mathrm{PDF}_0 \right] R_{jk}
\end{equation}

\subsubsection{Bayesian Reweighting of PDF sets}

Once PDF probability distributions are available as inputs, they can be updated to incorporate the new data. This is achieved by applying weights to the PDF probability distributions such that the prediction for observable $\langle\mathcal{O}(\mathrm{PDF})\rangle$ from equation \ref{eq:meanReplicas} changes to:

\begin{equation}
 \langle\mathcal{O}^{\mathrm{new}}(\mathrm{PDF})\rangle = \frac{1}{N_{\mathrm{rep}}} \sum_{k=1}^{N_{\mathrm{rep}}} w_k \mathcal{O}(\mathrm{PDF}_k)
\end{equation}

The weights $w_k$ calculated are here according to:

\begin{equation}
 w_k = \frac{(\chi^2_k)^{\frac{1}{2} (N_{\mathrm{data}}-1) } \exp^{-\frac{1}{2}\chi^2_k}}{ \frac{1}{N_{\mathrm{rep}}} \sum^{N_{\mathrm{rep}}}_{k=1}(\chi^2_k)^{\frac{1}{2}(N_{\mathrm{data}}-1)} \exp^{-\frac{1}{2}\chi^2_k}  },
\end{equation}

where $N_{\mathrm{data}}$ is the number of new data points, $k$ denotes the specific replica for which the weight is calculated and $\chi^2_k$ is between a given data point $y_i$ and its theoretical prediction obtained with the $k$-th PDF replica:

\begin{equation}
 \chi^2 (y,\mathrm{PDF}_k) = \sum_{i,j=0}^{N_{\mathrm{data}}} (y_i - y_i(\mathrm{PDF}_k)) \sigma^{-1}_{ij} (y_j-y_j(\mathrm{PDF}_k))  
\end{equation}

The weighted PDF probability distribution can be turned into a new ensemble of PDF replicas, based on which predictions for any observable can be calculated. This new, reweighted PDF probability distribution commonly is chosen to be based upon a smaller number of PDF sets compared to the input PDF probability distribution, because replicas that are incompatible with the data are discarded 
in order to create a more stream-lined PDF set.

\subsection{Profiling}

The impact of a new data set on a given PDF set can be quantitatively
estimated with a profiling procedure~\cite{Paukkunen:2014zia}. The
profiling is performed using a $\chi^2$ function which includes both
the experimental uncertainties and the theoretical uncertainties
arising from PDF variations:

\begin{eqnarray}
\nonumber \lefteqn{\chi^2(\boldsymbol{\beta_{\rm exp}},\boldsymbol{\beta_{\rm th}}) = }  \\ 
&& \nonumber \sum_{i=1}^{N_{\rm data}} \frac{\textstyle \left( \sigma^{\rm exp}_i + \sum_j \Gamma^{\rm exp}_{ij} \beta_{j,\rm exp} - \sigma^{\rm th}_i - \sum_k \Gamma^{\rm th}_{ik}\beta_{k,\rm th} \right)^2}{\Delta_i^2} \\
&&   + \sum_j \beta_{j,\rm exp}^2 + \sum_k \beta_{k,\rm th}^2\,.   \label{eq:chi2prof}
\end{eqnarray}
The correlated experimental and theoretical uncertainties are included
using the nuisance parameter vectors $\boldsymbol{\beta_{\rm exp}}$
and $\boldsymbol{\beta_{\rm th}}$, respectively. Their influence on
the data and theory predictions is described by the $\Gamma^{\rm
exp}_{ij}$ and $\Gamma^{\rm th}_{ik}$ matrices. The index $i$ runs
over all $N_{\rm data}$ data points, whereas the index $j$ ($k$)
corresponds to the experimental (theoretical) uncertainty nuisance
parameters. The measurements and the uncorrelated experimental
uncertainties are given by $\sigma^{\rm exp}_i$ and $\Delta_i$\,, respectively, and
the theory predictions are $\sigma_i^{\rm th}$. The $\chi^2$ function of
Eq.~\ref{eq:chi2prof} can be generalised to account for asymmetric PDF
uncertainties:

\begin{eqnarray}
   \Gamma^{\rm th}_{ik} \to \Gamma^{\rm th}_{ik} +  \Omega^{\rm th}_{ik}\beta_{k, \rm th}\,, \label{eq:iter}
\end{eqnarray}
where $\Gamma^{\rm th}_{ik} = 0.5(\Gamma^{\rm th+}_{ik} - \Gamma^{\rm
th-}_{ik})$ and $\Omega^{\rm th}_{ik} = 0.5(\Gamma^{\rm th+}_{ik}
+ \Gamma^{\rm th-}_{ik})$ are determined from the shifts of
predictions corresponding to up ($\Gamma^{\rm th+}_{ik}$) and down
($ \Gamma^{\rm th-}_{ik}$) PDF uncertainty eigenvectors.

The minimisation of Eq.~\ref{eq:chi2prof} in its original form leads
to a system of linear equations. The generalised function, with
asymmetric PDF uncertainties, is minimised iteratively: the values of $\Gamma^{\rm
th+}_{ik}$ are updated using $\beta_{k, \rm th}$ from the
previous iteration and following the substitution of Eq.~\ref{eq:iter}.
Several iterations are required to converge, and the procedure is verified
using the MINUIT program which yields identical results.

The value at the minimum of the $\chi^2$ function provides a
compatibility test of the data and theory.
In addition, the values at the minimum of the nuisance parameters
$\beta^{\rm min}_{k,\rm th}$ can be interpreted as optimisation
(``profiling'') of PDFs to describe the data~\cite{Paukkunen:2014zia}. Explicitly, the profiled
central PDF set $f'_0$ is given by

%\begin{small}
\begin{equation}
   f'_0 = f_0 + \sum_k  \beta^{\rm min}_{k, {\rm
   th}} \left( \frac{f^{+}_k - f^{-}_k}{2}  - \beta^{\rm min}_{k, {\rm th}}  \frac{f^{+}_k + f^{-}_k - 2f_0}{2} \right)\,, \label{eq:prof}
\end{equation}
%\end{small}
where $f_0$ is the original central PDF set and $f^{\pm}_k$ represents the
eigenvector sets corresponding to up and down variations.

The shifted PDFs have reduced uncertainties. In general, the
shifted eigenvectors are no longer orthogonal, but can be transformed
to an orthogonal representation using a standard diagonalisation
procedure, as in Ref.~\cite{Aaron:2009bp}.
In this method the covariance matrix $C$ of the PDF nuisance
parameters is diagonalised as

\begin{eqnarray}
  \lefteqn{\boldsymbol{\beta_{\rm th}^T} C \boldsymbol{\beta_{\rm th}} = \boldsymbol{\beta_{\rm th}^T} G^T D G \boldsymbol{\beta_{\rm th}} = 
  \boldsymbol{\beta_{\rm th}^T} (\sqrt{D} G)^T \sqrt{D}G \boldsymbol{\beta_{\rm th}}}  \nonumber \\
&&  =  (G' \boldsymbol{\beta_{\rm th}})^T  G'\boldsymbol{\beta_{\rm th}} = \boldsymbol{(\beta'_{\rm th})^T}  \boldsymbol{\beta'_{\rm th}}\,,
\label{eq:diag}
\end{eqnarray} 
where $G$ is an orthogonal matrix, $D$ is a positive definite diagonal
matrix, and $\sqrt{D}$ is a diagonal matrix built of
$\sqrt{D_{ii}}$. The matrices $G$ and $D$ can be constructed using
the eigenvectors and eigenvalues of the matrix $C$. The transformation
$G'$ can be adjusted, using orthogonal transformations, to keep the
new eigenvector basis aligned along the original as much as
possible. As a result of this adjustment, the transformation matrix
can take a triangular form with all diagonal elements greater than
zero.

The method can be extended to PDF sets with asymmetric
uncertainties: the transformation matrix is
determined using symmetrised uncertainties as in
Eq.~\ref{eq:diag}, and the orthogonal up and down PDF eigenvectors
$f^{+'}_i$ and $f^{-'}_i$ are calculated as

\begin{small}
\begin{eqnarray}
\nonumber   f^{+'}_i =& f'_0 + \sum_j G'_{ji} \left( \frac{f^{+}_j - f^{-}_j}{2} + G'_{ji} \frac{f^{+}_j + f^{-}_j - 2 f_0}{2}\right),  \\
\nonumber   f^{-'}_i =& f'_0 - \sum_j G'_{ji} \left( \frac{f^{+}_j - f^{-}_j}{2} - G'_{ji} \frac{f^{+}_j + f^{-}_j - 2 f_0}{2}\right).  
\end{eqnarray} 
\end{small}

\subsection{Usage of the alternative approaches in the \fitter framework}
 
The \fitter framework can be used to apply either PDF reweighting for
NNPDF-style PDF probability distributions as well as for PDF profiling for PDF sets with Hessian PDF eigenvector error sets. The \fitter will automatically determine which method to use based on the specified input PDF. 

This requires that the {\tt LHAPDF} module is installed, see sections \ref{sec:install_lhapdf}. In the \fitter steering files, as {\tt RunningMode} the following parameter has to be chosen: {\tt 'LHAPDF Analysis'}. This will write out the following files into the {\tt output} directory:

\begin{itemize}
 \item \textbf{NNPDF-style PDFs}:
 \begin{itemize}
  \item {\tt pdf\_BAYweights.dat}
  \item {\tt pdf\_GKweights.dat}
 \end{itemize}
  \item \textbf{Hessian PDFs}:
 \begin{itemize}
   \item {\tt{pdf\_vector\_cor.dat}}
   \item {\tt{pdf\_shifts.dat}}
   \item {\tt{pdf\_rotate.dat}}  
% -rw-rw-r-- 1 kristin kristin 5,4K Okt 23 16:45 Results.txt
% -rw-rw-r-- 1 kristin kristin 5,7K Okt 23 16:45 fittedresults.txt
\end{itemize}
\end{itemize}


These files can be used to either perform reweighting or profiling as described in the following. The module {\tt tools/PostProcess} is used for this purpose. Therefore the user has to change into this directory for further steps. 

\subsubsection{{\tt PostProcess}: Bayesian Reweighting}

The following steps need to be taken to reweight PDF sets after the main code has been run:

\textbf{./postproc reweight
usage: postproc reweight number\_output\_replicas pdf\_weights pdf\_dir\_in pdf\_dir\_out}

here, the following parameters need to be given:
\begin{itemize}
\item {\tt number\_output\_replicas } is the number of PDF sets that the replica should contain after the reweighting. 
\item {\tt pdf\_weights  } is the file with the weights (the output of the main part of the \fitter so either {\tt pdf\_BAYweights.dat} or {\tt pdf\_GKweights.dat}, depending on which weighting scheme you want to use.
\item {\tt pdf\_dir\_in }is the directory of the input PDF set, e.g. {\tt /share/LHAPDF/NNPDF30\_nlo\_as\_0118}
\item {\tt pdf\_dir\_out }is the directory of the input PDF set, e.g. {\tt /share/LHAPDF/NNPDF30\_myNewPDFset}
\end{itemize}

Eventually, some check histograms will be produced.

\subsubsection{{\tt PostProcess}: Profiling}


