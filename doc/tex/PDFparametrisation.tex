
\label{sec:pdfparam}
%%%%%%%%%%%
Different parametristion for PDFs at the starting scale present in \fitter\ are described in this section.
It starts from various standard functional forms (\ref{sec:stdform}), it follows with the bi-log-normal functional form (\ref{sec:log}), and it extends to more exotic forms based on generalised polynomial such as Chebyshev (\ref{sec:cheb}).

\subsubsection{Standard Functional form}
\label{sec:stdform}
%%%%
Through standard functional form it is undertstood a simple polynomial 
that interpolates between the low and high $x$ regions:
\begin{equation}
 xf(x) = A x^{B} (1-x)^{C} P_i(x),
\label{eqn:pdf_std}
\end{equation}
We identify few standard forms commonly used by PDF groups.

%%%%
\begin{description}
\item \bf{CTEQ style}\rm

%%%%
The notation used throughout this text reflects the 
notation used in the code.

\begin{equation}
 xf(x) = a_0 x^{(a_1+n)} (1-x)^{a_2} e^{a_3x} (1 + e^{a_4 x} + e^{a_5 x^2}),
\label{eqn:pdf_cteq}
\end{equation}
%
%%%%
\item \bf{HERAPDF style}\rm

%\subsubsection
%%%%
 The parametrised PDFs at HERA are the valence distributions
 $xu_v$ and  $xd_v$,  the gluon distribution $xg$, and the $u$-type and $d$-type 
$x\bar{U}$, $x\bar{D}$, where $x\bar{U} = x\bar{u}$, 
$x\bar{D} = x\bar{d} +x\bar{s}$. 
The following standard functional form is used to parametrise them
\begin{equation}
 xf(x) = A x^{B} (1-x)^{C} (1 + D x + E x^2),
\label{eqn:pdf}
\end{equation}
%
where the normalisation parameters, $A_{uv}, A_{dv}, A_g$,  are constrained by  
the QCD sum-rules, such that the counting  and  momentum conservation are preserved.
The $B$ parameters  $B_{\bar{U}}$ and $B_{\bar{D}}$ are set equal,
 $B_{\bar{U}}=B_{\bar{D}}$, such that 
there is a single $B$ parameter for the sea distributions. 
%
The strange quark distribution 
is already present at the starting scale and 
%
it is  assumed here that 
$x\bar{s}= f_s  x\bar{D}$ at $Q^2_0$. 
The  strange fraction is chosen to be $f_s=0.31$ which is
consistent with determinations 
of this fraction using neutrino induced di-muon production. 
%
In addition, to ensure that $x\bar{u} \to x\bar{d}$ 
as $x \to 0$,  
$A_{\bar{U}}=A_{\bar{D}} (1-f_s)$.
%
The $D$ and $E$ are introduced one by one until no further improvement in $\chi^2$ is found.
For the case when adding more precision data in the fit, as when adding HERA II data, this allows then for use of a more flexible parametrisation for the gluon and valence especially.
The best fit  results in a total of 10 free parameters when performing fits to solely HERA I data (fits are refered then ro as HERAPDF1.0), and of 13 free parameters when adding preliminary HERA II data on top (fits are refered then to as HERAPDF1.5).
\item \bf{Flexible style}\rm
%%%%

Flexible style is the extension of the ``HERAPDF style'' by allowing  extra $2$ free parameters for every PDF distribution, namely the $D$ and $E$ parameters for the medium $x$ region. It can be used to study the data sensitivity to PDFs. The total number of free parameters is therefore $22$.
\end{description}

\subsubsection{Bi-Log-Normal Functional Form}
\label{sec:log}
A bi-log-normal distribution is proposed by \cite{AndreSchoenig} to parameterise the x-dependence of the parton density function of the proton.
This new parameterisation
is motivated by arguments of multi-particle statistics. 
This function can be regarded as a generalisation of parameterisation commonly used by global fit groups.
The following parameterisation as general ansatz is proposed:
\begin{equation}
xf(x)=x^{p-b\log(x)}(1-x)^{q-\log(1-x)}.
\label{eq:AS}
\end{equation}
In order to satisfy the QCD sum rules this parametric form requires numerical integration.

\subsubsection{Chebyshev Polynomial Functional Form}
\label{sec:cheb}
A flexible Chebyshev polynomials based parameterisation is used for the gluon and sea densities. The polynomials
use $\log x$ as an argument to emphasise the low $x$ behaviour. 
The parameterisation is valid for $x>x_{min} = 1.7\times 10^{-5}$. The PDFs are multiplied
by $1-x$ to ensure that they vanish as $x\to 1$. The resulting parameterisation form is 
\begin{eqnarray}
x g(x) &=& A_g \left(1-x\right) \sum_{i=0}^{N_g-1} A_{g_i} T_i \left(-\frac{\textstyle 2\log x - \log x_{min} } {\textstyle \log x_{min} } \right)\,, \label{eq:glu} \\
x S(x) &=& \left(1-x\right) \sum_{i=0}^{N_S-1} A_{S_i} T_i \left(-\frac{\textstyle 2\log x - \log x_{min} } {\textstyle \log x_{min} } \right)\,. \label{eq:sea} 
\end{eqnarray}
Here the sum over $i$ runs up to $N_{g,S}=15$ order Chebyshev polynomials of the first type $T_i$ for
the gluon, $g$, and sea-quark, $S$, density, respectively. 
The normalisation $A_g$ is given by the momentum sum rule.
The advantages of the parameterisation given by equations~\ref{eq:glu},\ref{eq:sea} is that momentum
sum rule can be evaluated analytically and  already for $N \ge 5$ the fit quality
is similar to a standard Regge-inspired parameterisation with a similar number of parameters.

%==========================================
\subsubsection {Diffractive parametrization Functional Form}

\begin{description}
\item \bf{Pomeron parametrisation}\rm

\newcommand\AP {{\cal P}}

The Pomeron is parametrized at the initial
$Q_0^2$ in terms of two singlet distributions,
$f_{g}$ and $f_{+}$.
\begin{subequations}

\label{singlet}
\begin{eqnarray}
\frac{d}{dt}f_{+} &=&
\asotp\left[\AP_{\rm FF} f_{+} +\AP_{\rm FG}f_g
\right]
\\
\frac{d}{dt}f_{g} &=&
\asotp\left[
\AP_{\rm GF} f_{+} +\AP_{\rm GG}f_g
\right]
\end{eqnarray}
\end{subequations}

As $\Pom$ is neutral, $f_{q} = f_{\bar q}$ for each flavour $q$.
Assuming that all light quark PDFs are equal
\begin{equation}
f_d = f_u = f_s
\,,
\end{equation}
we have
\begin{subequations}
\label{eq:pm}
\begin{eqnarray}
f_{q-} &\equiv& 0
\\
f_{q+} &\equiv& 2 f_q
\end{eqnarray}
\end{subequations}

At $n_f = 3$
\begin{equation}
\label{eq:fq3}
f_{q+} = f_{+}/3,\; q = d,u,s
\,.
\end{equation}
% \ifFullVer
i.e.
% \begin{equation}
% \Sgl q = 0,\; q = d,u,s
% \,,
% \end{equation}
% where
\begin{equation}
\Sgl q \equiv f_{q+} - \frac{1}{n_f}f_{+}
 = 0,\;\mbox{for}\; q = d,u,s
\,.
\end{equation}
% \fi

This gives all PDFs for the FFNS, while for VFNS 
$f_{h+}$ for $h=c,b,t$ are generated dynamically above the respective
transition scales $Q_h^2$.
Hence at $n_f > 3$ the singlet has contributions from the heavy quarks
and we get non-trivial nonsinglet distributions $\Sgl{h}$ satisfying
\begin{equation}
\label{eq:nsevol}
\frac{d}{dt}\Sgl h = \asotp\, \AP_{(+)} \Sgl h
\end{equation}

\end{description}
%\subsection {Parametrization at \texorpdfstring{$Q_0^2$}{Q0}}
%\subsubsection {Parametrization at {$Q_0^2$}}
{\bf Parametrisation at {$Q_0^2$}} \\
%-----------------------------------------------------------
\label{sec:Par}

Full PDFs are given in analogy to Eq.~\ref{eq:FD3}
\begin{equation}
f_k^{D(3)}(\beta,Q^2,\xi) =
\hat\Phi_\Pom(\xi)\, f^\Pom_k(\beta,Q^2)
+
\Phi_\Reg(\xi)\, f^\Reg_k(\beta,Q^2)
\end{equation}
where $\hat\Phi_\Pom \equiv \Phi_\Pom/A_\Pom$,
with the fluxes given by Eq.~\ref{eq:intFlux} and Eq.~\ref{eq:flux}.

The Pomeron PDFs are parametrized as
\def\Cini#1#2{A^{(#1)}_#2}
\begin{equation}
\label{eq:fP0}
f^\Pom_N = \Cini N1  x^{\Cini N2} (1-x)^{\Cini N3}
%   \left(1 + \Cini N4 x \right)
  \; \exp\left(-\frac{d}{1.00001-x}\right)
\,,
\end{equation}
where the `dumping factor' $d$ is taken as 0.01 or 0.001.
$N = \mathrm G$ for gluon and $N = \mathrm S$ for `singlet': $f_{\rm S} \equiv f_+(n_f=3)$,
cf. Eq.~\ref{eq:fq3}.



% The Reggeon PDFs $f^\Reg_k$ are taken from pion.

%\subsubsection {HERAFitter parameters}






