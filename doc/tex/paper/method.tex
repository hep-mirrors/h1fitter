
The methodology employed by HERAFitter provides,  
 on one hand, faithful conditions adopted by different theoretical 
groups in extracting global PDF sets, and, on the other hand,  presents
a flexible framework to implement new theoretical developments for 
direct comparison.

The QCD fit formalism within HERAFitter gives access to different functional 
forms used to parametrise PDFs at the starting scale, different definitions for $\chi^2$ to account for systematic uncertainties in extraction of theory parameters, different treatement of experimental uncertainties, and alternative theoretical models. 

This elaborate project also addresses the performance related issues by optimising the time of calculations using and incorporating innovative techniques such as cache option, fast evolution kernels, grid techniques making the platform a practical engine for iterative usage.

As an alternative to a complete QCD fit, reweighing method to estimate impact of new data is provided. This has been already advocated by the NNPDF collaboration \cite{PDFreplicareweighing} and HERAFitter can provide direct access to this method. The method has been extended to work not only on the replica method, but also on the eigenvectors (as introduced by MSTW group \cite{PDFeigreweighing}).





\subsection{Chisquare representation}


\subsection{Treatment of Experimental Uncertainties}


\subsection{Treatment of Theoretical Input Parameters}

\subsection{Performance Optimisation}

The above mentioned features make HERAFitter a powerful project that encapsulates state of the art developments from struggles on reaching atmost experimental precision to the state of the art theory developments. 



