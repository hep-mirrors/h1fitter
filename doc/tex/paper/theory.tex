
\label{sec:theory}
\def\kt{\ensuremath{k_t}}
\newcommand{\Pmax}{p}
\newcommand{\CCFM}{CCFMa,CCFMb,Catani:1989sg,CCFMd}

The proton PDFs are classically extracted from QCD fits by a measure of 
the agreement between data and theory models.
The fit procedure used in the \fitter\ framework is common to all processes and it consists first 
in parametrising the PDFs at a starting scale  $Q^2_0$, chosen to be below the charm mass threshold. 
The PDFs are then evolved using coupled, integro-differential
Dokshitzer-Gribov-Lipatov-Altarelli-Parisi (DGLAP)~\cite{Gribov:1972ri,Gribov:1972rt,Lipatov:1974qm,
Dokshitzer:1977sg,Altarelli:1977zs} evolution equations 
as implemented in the QCDNUM~\cite{qcdnum} program in the $\overline{MS}$ scheme 
(LO, NLO nd NNLO evolutions are available~\cite{Curci:1980uw,Furmanski:1980cm}).
%The renormalisation and factorisation scales are set to $Q^2$. 
%(LO and NNLO evolutions are also available).
The PDFs calculated at a scale corresponding to a measured cross section are convoluted with the partonic 
cross sections to calculate the predicted cross section. For all measurement points, the predicted 
and measured cross sections together with their corresponding errors are used to build a global $\chi^2$, 
minimized to determine the initial PDF parameters. This generic procedure includes a number of 
subtleties and assumptions, depending on the measurements, scale and available calculations.

In following, the theoretical input for various processes is described,
