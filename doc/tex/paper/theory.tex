
\def\kt{\ensuremath{k_t}}
\newcommand{\Pmax}{p}
\newcommand{\CCFM}{CCFMa,CCFMb,Catani:1989sg,CCFMd}


%The proton PDFs are classically extracted from QCD fits by a measure of 
%the agreement between experimental data and corresponding theory models.
%During the fit procedure in the \fitter\ framework, the PDFs are
%parametrised at a starting scale $Q^2_0$ chosen to be below the charm 
%mass threshold and then evolved using coupled, integro-differential
%Dokshitzer-Gribov-Lipatov-Altarelli-Parisi 
%(DGLAP)~\cite{Gribov:1972ri,Gribov:1972rt,Lipatov:1974qm,
%Dokshitzer:1977sg,Altarelli:1977zs} evolution equations 
%as implemented in the QCDNUM~\cite{qcdnum} program in the $\overline{\text{MS}}$ scheme. 
%The evolution can be performed in the LO, NLO or NNLO accuracy~\cite{Curci:1980uw,Furmanski:1980cm}.
%
%In following, the theoretical models for various processes available in \fitter\ are described.
The PDFs are determined by measuring 
the agreement between experimental data and corresponding theory models within 
the DGLAP~\cite{Gribov:1972ri,Gribov:1972rt,Lipatov:1974qm,
Dokshitzer:1977sg,Altarelli:1977zs} formalism.
Models which are available in \fitter\ for various processes are described in the following text.


\subsection{Deep Inelastic Scattering Formalism and Schemes}
\label{dissection}


DIS data provide the tightest constraints on the PDFs so far.
Deep inelastic scattering is the lepton scattering on the 
constituents of the proton by a virtual exchange of a neutral (NC) 
or charged (CC) boson and, as a result, a scattered lepton and a 
multihadronic final state are produced.
The DIS kinematic variables are the negative squared four-momentum of 
the exchange boson, $Q^2$, the 
scaling variable $x$, which can be related in the parton model to 
the fraction of momentum carried by the struck quark, and the 
inelasticity parameter $y$, which is the fraction of the energy 
transferred to the hadronic vertex. 
\\
%
The NC (and similarly CC) cross section can be expressed in terms of structure functions:
\begin{eqnarray}
%  \nonumber
   \frac{d^2\sigma_{NC}^{e^{\pm} p}}{dxdQ^2}=\frac{2\pi\alpha^2}{xQ^4} 
     \big [ Y_{+} \tilde F_2^{\pm} \mp Y_{-}x \tilde F_3^{\pm} - y^2 \tilde F_L^{\pm} \big ],
 %\label{eq:NC not reduced}
\end{eqnarray}
where $Y_{\pm} = 1 \pm (1-y)^2$ with $y$ being the inelasticity. The structure function $\tilde F_2$
is the dominant contribution to the cross section, $x \tilde F_3$ is important at high $Q^2$ and $\tilde F_L$ is sizable 
only at high $y$. 
In the framework of perturbative QCD the structure functions are directly related to the 
parton distribution functions, i.e. in leading order (LO)  $F_2$ is the momentum sum of quark and anti-quark distributions, 
$F_2 \approx x \sum e^2_q (q+ \overline q)$, and $xF_3$ is related to their difference, 
    $xF_3 \approx x \sum 2e_q a_q (q- \overline q)$. At higher orders, terms related to the gluon density distribution
($\alpha_s g$) appear.
\\
In analogy to neutral currents, the inclusive CC $ep$ cross section can be expressed 
in terms of structure functions and in LO the $e^+p$ and $e^-p$ cross sections are sensitive to different quark 
densities:
\begin{eqnarray}
%  \nonumber
    \begin{array}{rll}
   e^{+}:  & & \tilde \sigma_{CC}^{e^{+} p} = 
                x[\overline u +\overline c] + (1-y)^2 x[ d+s ]  \\
   e^{-}:  & & \tilde \sigma_{CC}^{e^{-} p} = 
                x[ u +c] + (1-y)^2 x[\overline d +\overline s ].
    \end{array}
\end{eqnarray}
%
The QCD predictions for the DIS structure functions are obtained by convoluting 
the PDFs with the coefficient functions calculated using various schemes, i.e.
the general mass Variable-Flavour number (GM-VFN)~\cite{VFN} schemes or the
Fixed-Flavour number (FFN)~\cite{Laenen:1992, Laenen:1993, Riem:1995}. 
%\fitter\ implements the zero mass variable flavour number (ZMVFN) scheme
% from QCDNUM and are discussed in the following subsections.
%In the FFN scheme, heavy quark contributions are explicitly included in the 
%hard cross sections, . 
%In the VFN scheme, PDFs corresponding to heavy quarks are introduced and the number of active 
%flavors changes by one unit when the scale crosses the threshold for heavy quark distribution ($Q^2 > m_{Q}^2$).
%The various treatments for the heavy quark thresholds are implemented as provided by the MSTW group
%
%The evolution program QCDNUM~\cite{qcdnum} used in \fitter\ provides 
%the calculations of the deep inelastic structure functions in the zero-mass, 
%generalised mass and the fixed flavour number schemes. 
\\
The following VFN schemes with various treatments for the heavy 
quark thresholds are considered in \fitter\ :
The Thorne Roberts (TR) scheme with its variants at NLO and
NNLO~\cite{Thorne:1997ga,Thorne:2006qt,Martin:epC63,Thorne:6180} 
as provided by the MSTW group,
the ACOT scheme with its variants at LO and NLO as provided by the CTEQ group. 
In addition, the zero-mass variable flavour number scheme (ZM-VFNS) where 
heavy quark densities are included in the proton for $Q^2>>m_h^2$ but are treated 
as massless in both the initial and final states can be used in \fitter\ .
The FFN scheme is available via the QCDNUM implementation and via the 
{\tt OPENQCDRAD}~\cite{openqcdrad:page} interface.
Each of these schemes is briefly discussed below.

\begin{description}
\item \bf {GM-VFN Thorne-Roberts scheme:} \rm
%\subsubsection{GM-VFN Thorne-Roberts scheme}
%
The Thorne-Roberts (TR) scheme smoothly connect the two regions: 
scales below ($Q^2<m_h^2$) and scales much above the heavy quark scale threshold ($Q^2>>m_h^2$). 
%However, the connection is not unique.
%A GM-VFNS can be defined by demanding equivalence of the $n_f = n$ (FFNS) 
%and $n_f = n+1$ flavour (ZM-VFNS) descriptions above the transition point for the new parton distributions
%(they are by definition identical below this point), at all orders.
There are two different variants of the TR schemes: TR standard (as used in MSTW PDF 
sets~\cite{Thorne:1997ga,Thorne:2006qt,Martin:epC63}) 
and TR optimal~\cite{Thorne:6180}, with a smoother transition across the heavy quark mass scales 
and both of them are accessible within the \fitter\ package.
The calculations are available to NLO and NNLO.  

%%%%
\vspace{0.1cm}
\item \bf {GM-VFN ACOT scheme:} \rm

The Aivazis-Collins-Olness-Tung scheme belongs to the group of VFN 
factorisation schemes that use the renormalization method of 
Collins-Wilczek-Zee (CWZ) \cite{CWZ}.
This scheme involves a mixture of the $\overline{\text{MS}}$ scheme 
for light and heavy (when the factorisation scale is larger than the heavy quark mass) partons
and the zero-momentum subtraction renormalisation scheme for graphs with heavy quark lines 
(if the factorisation scale is smaller than the mass of the heavy quark threshold). 
% is this sentence below is important?
%The DGLAP kernels and PDF evolution are pure $\overline{\text{MS}}$, 
%therefore, the ACOT scheme is considered to be a minimal extension of the $\overline{\text{MS}}$ scheme.

Within the ACOT package, different variants of the ACOT scheme are available:
ACOT-Full, S-ACOT-$\chi$, ACOT-ZM, $\overline{\text{MS}}$ at LO and NLO. 
For the longitudinal structure function higher order calculations are also available. 
The ACOT-Full implementation takes into account the quark masses 
and it reduces to ZM $\overline{\text{MS}}$ scheme in the limit of masses going to zero, 
but it has the disadvantage of being quite slow.

\vspace{0.1cm}
\item \bf {FFN schemes:} \rm

In the FFN scheme only the gluon and the light quarks are considered
as partons within the proton and massive quarks are produced perturbatively in the final state.
In \fitter\ this scheme can be accessed via the 
QCDNUM implementation or interface to the open-source code OPENQCDRAD (ABM)~\cite{openqcdrad:page}.
The later implementation also includes the running mass definition of the heavy quark 
mass ~\cite{Alekhin:runm}.
This scheme has the advantage of reducing the sensitivity of the DIS cross sections to
higher order corrections, and improving the theoretical precision of the mass definition. 
In QCDNUM, the calculation of the heavy quark contributions to DIS structure functions
are available at NLO and only electromagnetic exchange contributions are taken into account.
In the ABM implementation, the QCD corrections to the massive Wilson coefficients 
up to the currently best known approximate NNLO for the NC heavy-quark 
production~\cite{SMoch:npb864} and up to NLO for the CC case are available.
\end{description}

%%%%%%%%%%%
%\item \bf {Electroweak corrections for \texorpdfstring{$ep$}{ep} scattering:} \rm
The calculations of higher-order electroweak corrections to DIS scattering at 
HERA are performed in the on-shell scheme where the gauge bosons masses $M_W$ and 
$M_Z$ are treated symmetrically as basic parameters together with the top, Higgs 
and fermion masses.
\\
In the \fitter\, the electroweak corrections 
for the DIS process are based on the EPRC package~\cite{SpiesbergerPrivComm}.
The code provides the running of $\alpha$ using the most recent parametrisation
of the hadronic contribution to $\Delta_\alpha$ \cite{Jegerlehner}, as well as 
an older one from Burkhard \cite{Burkhard}.



\subsection{Drell Yan processes  in $pp$ or $p\bar p$ collisions}
\label{dysection}

%This section presents calculations of Drell Yan processes that can be used to 
%predict lepton pair production at the LHC or Tevatron.
The Drell Yan (DY) process constrain all different quark combinations
providing valuable information about PDFs.
In $pp$ and $p\bar p$ scattering, the Z$/\gamma\*$ and W production 
probe bi-linear combinations of quarks. 
The complementary information on the different quark densities
can be obtained from W asymmetry ($d$ and $u$, also their ratio),
the ratio of the W and Z cross sections (sensitive to the favor 
composition of the quark sea), associated W and Z production with
heavy quarks (provides an access to $s$ and $c$ quark densities).
%

Presently, the calculations of the W and Z processes are known for many observables up to 
NNLO order, for example, MCFM~\cite{Campbell:1999ah,Campbell:2000je,Campbell:2010ff} 
package is available for NLO calculations,
FEWZ~\cite{FEWZ} and DYNNLO~\cite{DYNNLO} for NLO and NNLO.
There are several possibilities for obtaining the theoretical
predictions for the DY production in \fitter: the LO triple 
differential cross sections calculated withing the package or,
since the DY production theory calculaton is highly demanding
in terms of the computing power and time, the fast grid techniques 
can be use as an alternative (see section~\ref{sec:techniques}
for details).




%The most abundant processes at the LHC are the production of the $W$ and $Z$ bosons, therefore measurements of the $W$ and $Z$ cross-sections are very precise. Their  LO decomposition in terms of quark distributions show strong 


%Alternatively, one can obtain the NLO predictions directly by using 
%APPLGRID or FASTNLO techniques, which rely on the factorisation theorem by 
%decoupling the hard scattering coefficients from PDFs.
%The hard scattering coefficients are calculated once and stored into a grid 
%for a given kinematic bin, speeding up the convolution process with the PDFs 
%and thus allowing to for fast QCD fits. 


%These methods are described in more detail in section \ref{sec:theory:jets}.
%An independent treatment for the electro-weak corrections is applied as the 
%independent k-factors, using packages such as SANC and FEWZ.

\subsection{Jet production in $ep$ and $pp$ collisions}
\label{jetsection}
%In this subsection, the use of the factorisation formalism is fully exploited for the 
%calculations of the inclusive jets and dijet cross sections.
%This sections presents various fast calculational techniques for jet production based on
%the factorization formalism.

Similarly to DY case, the calculation of higher order jet cross sections 
is very demanding in terms of computing power. 
%Although the concept is process independent, 
%The perturbative 
%coefficients are calculated by the \texttt{NLOJET++}, which is 
Therefore, to allow the possibility to include the $ep$, $pp$ or $p\bar p$ 
jet cross section 
measurements in QCD fits to extract PDF and $\alpha_s$ fits, the perturbative
coefficients have to be pre-computed in a PDF and $\alpha_s$ 
independent way. For this reason, the fast grid tools to the theory calculations
obtained with MCFM~\cite{Campbell:1999ah,Campbell:2000je,Campbell:2010ff} and NLOJET++~\cite{Nagy:1998bb,Nagy:2001fj}, 
which are interfaced to the \fitter , are also exploited for the jet production (see section~\ref{sec:techniques}). 


\subsection{Cross Sections for \texorpdfstring{$t\bar{t}$}{t-tbar} production in $pp$ or $p\bar p$ collisions}
%This section presents the calculation tool available 
%in \fitter\ for top-guark pair prouction in $p \bar{p}$ and $pp$ collisions.
%
Top-quark pairs ($t\bar{t}$) are mainly produced at hadron colliders via $gg$ fusion and
$q \bar q$ annihilation thus providing possibility to constrain the gluon density in the proton. 
%There are also $q q'$ and $q g$ production modes.
In \fitter\ the program HATHOR~\cite{Aliev:2010zk} is interfaced which allows the calculation of 
the expected total $t \bar t$ cross section at 
$p \bar p$ and $p p$ colliders up to approximate NNLO accuracy.
Version 1.3 of HATHOR includes the exact NNLO for $q \bar q \to t \bar t$ \cite{Baernreuther:2012ws}
as well as a new high-energy constraint on the approximate NNLO calculation obtained from
soft-gluon resummation \cite{Moch:2012mk}.
The default choice for renormalization and factorization scale in $t \bar t$ production is the top-quark mass, $m_t$.
The pole mass scheme is typically employed for $m_t$ but HATHOR also supports calculations in
the $\overline{\text{MS}}$ scheme.



