The \fitter program has been used in a number of experimental and theoretical analyses. 
This list includes several LHC analyses of SM processes, namely
inclusive Drell-Yan and $W$and $Z$ production~\cite{atlas:strange,cms:strange,atlas:hm,Aad:2014qja,Aad:2014xca}, and
inclusive jet production \cite{atlas:jets}.
The results of QCD analyses using \fitter were also
published by HERA experiments for inclusive \cite{h1zeus:2009wt,h1:2012kk} and
 heavy flavour production measurements \cite{h1zeus:charm, Abramowicz:2014zub}.
The following phenomenological studies have been performed with \fitter:
a determination of the transverse momentum dependent gluon density using precision HERA data \cite{Hautmann:2013tba}, 
an analysis of HERA data within a dipole model \cite{Luszczak:2013rxa},
the study of the low-x uncertainties in PDFs determined from the HERA data using 
different parametrisations \cite{Chebyshev} and the impact of QED radiative corrections on PDFs \cite{Sadykov:2014aua}.
A recent study based on a set of PDFs determined with the \fitter and addressing 
the correlated uncertainties between different orders has been published in \cite{hfcorrpaper}. 

The \fitter framework has been used to produce PDF grids from QCD analyses performed at 
HERA \cite{h1zeus:2009wt,hera:grids} and at the LHC \cite{atlas:grids}, using 
measurements from ATLAS~\cite{atlas:strange,atlas:jets}. These PDFs can be used to study predictions for SM 
or beyond SM processes. Furthermore, \fitter provides the possibility to perform various benchmarking exercises
\cite{Butterworth:2014efa} and impact studies for possible future colliders
as demonstrated by QCD studies at the LHeC~\cite{lhec:studies}.









