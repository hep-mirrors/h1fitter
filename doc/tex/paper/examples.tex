\fitter\ has been successfully integrated in the high energy community as a much needed means to provide understanding and interpretation of new measurements in the context of QCD theory, a field limited by the precision of the PDFs.  
The \fitter\ platform not only allows the extraction of PDFs but also of theory parameters such as the strong coupling and heavy quark masses. The parameters and distributions are ouput with a
quantitative asssessment of the fit quality with fully detailed information on 
experimental and theoretical uncertainties.
The results are also output to PDF grids that can be used to study predictions for SM or beyond SM processes, as well as for the study of the impact of 
future collider measurements (using pseudo-data).


So far the \fitter\ platform has been used to produce grids 
from the QCD analyses performed at 
HERA (HERAPDF series~\cite{h1zeus:2009wt}), and their extension to the LHC 
using 
measurements from ATLAS~\cite{atlas:strange,atlas:jets} (the first ever ATLAS PDF sets~\cite{atlas:grids}).
% to the QCD studies performed using pseudo-data to study the potential of possible future collider design such as LHeC.

New results that have been based on the \fitter\ platform include 
the follwing SM processes 
studied at the LHC:  inclusive Drell-Yan and $W$and $Z$ 
production~\cite{atlas:strange,atlas:hm,cms:strange};
inclusive jets~\cite{atlas:jets,cms:jets} production.
% and top measurements{\bf you need a reference for the top studies}
At HERA, the results of QCD analyses using \fitter\ are 
published for inclusive H1 measurements~\cite{h1:2012kk}
and the recent combination of charm production measurements 
in DIS~\cite{h1zeus:charm}.
The \fitter\ framework also provides an unique possibility to 
make impact studies for future colliders
as illustrated by the QCD studies that have been performed to 
explore the potential of the LHeC data~\cite{lhec:studies}.

In addition, a recent study based on a set of parton distribution functions 
determined with \fitter\ program using HERA data was performed~\cite{hfcorrpaper}.
It addresses the issue of correlations between uncertainties for the LO,
NLO and NNLO sets. These sets are then propagated to study uncertainties 
for ratios of cross sections calculated at different order in QCD and  
a reduction of overall theoretical uncertainty is observed.

