

%\subsection{Alternative to DGLAP DIS models}
QCD calculations based on the DGLAP~\cite{Gribov:1972ri,Gribov:1972rt,Lipatov:1974qm,
Dokshitzer:1977sg,Altarelli:1977zs} evolution equations are very successful in describing
all relevant hard scattering data in the perturbative region $Q^2 \gtrsim 1 \GeV^2$.
At small-$x$ and small-$Q^2$ DGLAP dynamics may be modified by non-perturbative QCD 
effects like saturation-based dipole models and other higher twist effects.
%
Different approaches that are alternatives to the DGLAP formalism can be used to analyse DIS data in \fitter.
These include several different dipole models and the use of 
transverse momentum dependent, or unintegrated PDFs (uPDFs).

\subsection{Dipole Models}

The dipole picture provides an alternative approach to proton-virtual photon
 scattering at low $x$ which can be applied to both inclusive and 
diffractive processes.
 In this approach, the virtual photon fluctuates into a $q\bar q$ (or $q\bar q g$) 
 dipole which interacts with the proton~\cite{NNZ:91}.  
The dipoles can be considered as quasi-stable quantum mechanical states, which have very long 
life time $\propto 1/m_p x\;$ and a size which is not changed by scattering with the proton.
The dynamics of the interaction are embedded in a dipole scattering amplitude.

Several dipole models which assume different behaviour of the dipole-proton 
cross section are implemented in \fitter:
the Golec-Biernat-W\"usthoff (GBW)
dipole saturation model~\cite{Golec-Biernat:1998js},
a modified GBW model which takes into account the effects of  
DGLAP evolution termed the Bartels-Golec-Kowalski (BGK) dipole model~\cite{Bartels:2002cj}
and the colour glass condensate approach to the high parton density 
regime termed the Iancu-Itakura-Munier (IIM) dipole model~\cite{Iancu:2003ge}.
%\end{itemize}

%\begin{description}
\paragraph{GBW model:} \rm
In the GBW model the dipole-proton cross section $\sigma_{\text{dip}}$ is given by
\begin{equation}
\label{eGBW}
   \sigma_{\text{dip}}(x,r^{2}) = \sigma_{0} \left(1 - \exp \left[-\frac{r^{2}}{4R_{0}^{2}(x)} \right]\right),
\end{equation}
where $r$ corresponds to the transverse separation between the quark and the antiquark, and $R_{0}^{2}$
 is an $x$-dependent scale parameter which represents the spacing of the gluons in the proton. 
$R_{0}^{2}$ takes the form, $R_0^2(x) = (x/x_0)^\lambda  1/ {\rm \GeV}^{2}$, and is called the saturation radius.
The cross-section normalisation $\sigma_0$, $x_0$, and $\lambda$ are parameters 
of the model commonly fitted to the DIS data.
This model gives exact Bjorken scaling when the dipole size $r$ is small.
%%%%
 
\paragraph{BGK model:} \rm
The BGK model is a modification of the GBW model assuming that the
spacing $R_0$ is inverse of the gluon density and taking
into account the DGLAP evolution of the latter.
The gluon density parametrised at some starting scale by Eq.~\ref{eqn:pdf_std}
is evolved to larger scales using DGLAP evolution.

\paragraph{BGK model with valence quarks:} \rm
The dipole models are valid in the low-$x$ region only, where the valence quark contribution to the total proton momentum 
is 5\% to 15\% for $x$ from 0.0001 to 0.01 \cite{Collaboration:2010ry}.
%, of the order of 5\%. 
The new HERA $F_2$ measurements have a precision which is better than 2$\%$. 
Therefore, in \fitter the contribution of the valence quarks can be taken into account~\cite{Luszczak:2013rxa}.

\paragraph{IIM model:} \rm
The IIM model assumes an expression for the dipole cross section which is based on the 
Balitsky-Kovchegov equation~\cite{Balitsky:1995ub}. The explicit formula for $\sigma_{\text{dip}}$ 
can be found in~\cite{Iancu:2003ge}. 
The alternative scale parameter $\tilde{R}$, $x_{0}$ and $\lambda$ are fitted parameters of the model.

\subsection{Transverse Momentum Dependent PDFs}


\def\kt{\ensuremath{k_t}}
\def\pt{\ensuremath{p_t}}


QCD calculations of multiple-scale processes  and complex final-states
can necessitate the use of transverse-momentum dependent (TMD)~\cite{Collins:2011zzd}, or 
unintegrated, parton distribution and parton decay 
functions~\cite{Aybat:2011zv,Buffing:2013eka,Buffing:2013kca,Buffing:2012sz,Mulders:2008tf,Jadach:2009gm,Hautmann:2009zzb,Hautmann:2012pf,Hautmann:2007gw}.   
TMD factorisation has been proven recently \cite{Collins:2011zzd} for inclusive DIS. TMD factorisation has also been proven in the high-energy (small-$x$) limit \cite{Catani:1990xk,Collins:1991ty,Hautmann:2010be} for 
particular hadron-hadron scattering processes, like heavy flavor, vector boson and Higgs production, 
  
In the framework of high-energy factorisation~\cite{Catani:1990xk,Catani:1990eg,Catani:1993ww} 
the DIS cross section can be written as a convolution in 
both longitudinal and transverse momenta of the TMD parton density function 
${\cal A}\left(x,\kt,\mu_{F}^2\right)$    
 with the off-shell parton scattering matrix elements, as follows 
\begin{equation}
 \sigma_j (x, Q^2) = \int_x^1  
d z \int d^2k_t \ 
\hat{\sigma}_j(x,Q^2,{z},k_t) \ 
 {\cal  A}\left( {z} ,\kt, \mu_{F}^2 \right) 
\label{kt-factorisation}
\end{equation}
with the DIS cross sections 
$\sigma_j$, ($j= 2 , L$) related to the  structure functions $F_2$ and $F_L$.
The hard-scattering kernels ${\hat \sigma}_j$ of Eq.~\ref{kt-factorisation},    are $k_t$-dependent and the evolution  of the 
transverse-momentum dependent gluon density 
${\cal A} $ is obtained by combining the resummation of small-$x$ logarithmic 
contributions \cite{Lipatov:1996ts,Fadin:1975cb,Balitsky:1978ic} with medium-$x$ and large-$x$ 
contributions to parton  splitting~\cite{Gribov:1972ri,Altarelli:1977zs,Dokshitzer:1977sg} according to the 
CCFM evolution equation~\cite{Ciafaloni:1987ur,Catani:1989sg,Marchesini:1994wr}.
  
The factorisation formula (\ref{kt-factorisation})  
allows resummation of logarithmically enhanced small-$x$ contributions  
to all orders in perturbation theory,  
both in the  hard scattering coefficients and 
in the parton evolution, fully taking into account the 
dependence on the factorisation scale $\mu_{F}$ and on the 
factorisation scheme~\cite{Catani:1994sq,Catani:1993rn}.  
 
The cross section $\sigma_j$, ($j= 2, L$) is calculated in a FFN scheme, using the boson-gluon fusion process ($\gamma^* g^* \to q \bar{q}$). The masses of the 
quarks are explicitly included as parameters of the model.
In addition to $\gamma^* g^* \to q\bar{q}$,  the contribution from valence quarks is included 
via $\gamma^* q \to q$ by using a CCFM evolution of 
valence quarks~\cite{Deak:2010gk,Hautmann:2013tba}. 

\paragraph{CCFM Grid Techniques:} \rm

The CCFM evolution cannot be written easily in an analytic closed form. For this 
reason a Monte Carlo method is employed, which is however time-consuming, and thus
cannot be used directly in a fit program. 

Following the  convolution method introduced in~\cite{Jung:2012hy,Hautmann:2013tba}, the 
kernel $ \tilde {\cal A}\left(x'',\kt,\Pmax\right) $ is determined from the Monte Carlo  solution of the CCFM evolution equation, 
and then folded with a non-perturbative starting distribution ${\cal A}_0 (x)$
{ 
\begin{eqnarray}
x {\cal A}(x,\kt,\Pmax) &= &x\int dx' \int dx'' {\cal A}_0 (x') \tilde{\cal A}\left(x'',\kt,\Pmax\right) 
 \delta(x' 
x'' - x) 
\nonumber  
\\
& = & \int dx' {{\cal A}_0 (x') }  
\frac{x}{x'} \ { \tilde{\cal A}\left(\frac{x}{x'},\kt,\Pmax\right) }, 
\end{eqnarray}
}
where $\kt$ denotes the transverse momentum of the propagator gluon and $\Pmax$ is the 
evolution variable.

The kernel $\tilde{\cal A}$ incorporates all of  the dynamics of the evolution.  
It is defined on a grid of $50\otimes50\otimes50$ bins in $ x, \kt, \Pmax$.  
The binning in the grid is logarithmic, except for the longitudinal variable 
$x$ for which 40 bins in logarithmic 
spacing below 0.1, and 10 bins in linear spacing above 0.1 are used.

Calculation of the cross section according to Eq.~\ref{kt-factorisation} involves a time-consuming
multidimensional Monte Carlo integration which suffers from numerical fluctuations.  
This cannot be employed directly in a fit procedure. Instead the following equation is applied:
\begin{eqnarray}
\sigma(x,Q^2) & = & \int_x^1 d x_g {\cal A}(x_g,\kt,\Pmax) \hat{ \sigma}(x,x_g,Q^2) 
\nonumber\\
  & = & \int_x^1 dx' {\cal A}_0 (x')  \tilde{ \sigma}(x/x',Q^2),
    \label{final-convolution}
 \end{eqnarray}
where first $ \tilde{ \sigma}(x',Q^2)$ is calculated numerically with a Monte Carlo integration 
on a grid in $x$ for the values of $Q^2$ used in the fit. Then the last step in Eq.~\ref{final-convolution}  
is performed with a fast numerical gauss integration, which can be used directly in the fit.

%\vspace{0.1cm}
\paragraph{Functional Forms for TMD parametrisation:} \rm

For the starting distribution ${\cal A}_0$, at the starting scale $Q_0^2$, 
the following form is used:
\begin{eqnarray}
x{\cal A}_0(x,\kt) &=& N x^{-B}(1 -x)^{C}\left(1 -D x 
+ E \sqrt{x}   \right) \nonumber\\
   &\times &\exp[ - k_t^2 / \sigma^2 ], 
\label{a0-5par}
\end{eqnarray}
where $ \sigma^2  =  Q_0^2 / 2 $ and $N, B, C, D, E$ are free parameters.
Valence quarks are treated using the method of Ref.~\cite{Deak:2010gk} as described 
in Ref. \cite{Hautmann:2013tba} with a starting distribution taken from any collinear PDF
and imposition of the flavor sum rule at every scale $p$.
\\
The TMD parton densities can be plotted either with \fitter provided tools 
or with \tt TMDplotter\rm~\cite{tmdlref}.

%\end{description}

