The constant inflow of new experimental measurements 
with unprecedented accuracy from hadron colliders is a remarkable challenge 
for the high energy physics community to provide higher-order theory 
predictions and to develop efficient tools and methods for data analysis.
The recent discovery of the Higgs boson \cite{Aad:2012tfa,Chatrchyan:2012ufa} 
and the extensive searches
for signals of new physics in LHC proton-proton collisions
demand high-precision computations to test the validity of the Standard Model (SM)
and factorisation in Quantum Chromodynamics (QCD).
According to the collinear factorisation in perturbative QCD (pQCD)
hadronic inclusive cross sections are written as
%
%The discovery of the Higgs boson~\cite{Aad:2012tfa,Chatrchyan:2012ufa}
%and extensive searches for signals of new physics at the LHC demands accurate precision of the Standard Model (SM) predictions for
%hard scattering processes in hadron-hadron collisions.
%\\
%The most common approach to calculate the SM cross sections for  
%such reactions is to use collinear factorisation in perturbative QCD (pQCD)~\cite{Collins:1989}:
%{\small
\begin{eqnarray}
\small
%\begin{array}{lcl}
%&&\sigma(h_1  h_2 \rightarrow l^{+} l^{-} + X)  = 
%\nonumber\\
%&&\sum\limits_{a,b}\,  \int_{x_1}^{1} d\xi_1 \int_{x_2}^{1} d\xi_2\, f_{h_1\rightarrow a}(\xi_1,\as(\mur),\mur,\muf) 
% f_{h_2\rightarrow b}(\xi_2,\as(\mur),\mur,\muf)\nonumber\\
%&&\times  \, \hat{\sigma}^{ab}(\frac{x_1}{\xi_1},\frac{x_2}{\xi_2};\as(\mur),Q,\muf,\mur) 
%+ {\cal O}\left(\frac{\Lambda^2}{Q^2}\right)\,,\\
%
%\sigma(\as(\mur),\mur,\muf)&=& \sum\limits_{a,b}\,  \int\limits_{0}\limits^{1} dx_1\ dx_2 \nonumber \\
%&\times& f_a(x_1,\as(\mur),\muf) 
% f_b(x_2,\as(\mur),\muf) \nonumber \\ 
%&\times&  \, \hat{\sigma}^{ab}(x_1,x_2;\as(\mur),\mur,\muf)\,
\sigma(\as(\mur^2),\mur^2,\muf^2)&=& \sum\limits_{a,b}\,  \int\limits_{0}\limits^{1} dx_1\ dx_2  f_a(x_1,\muf^2) f_b(x_2,\muf^2) \nonumber \\ 
&\times&  \, \hat{\sigma}^{ab}(x_1,x_2;\as(\mur^2),\mur^2,\muf^2),
%+{\cal O}\left(\frac{\Lambda^2}{Q^2}\right)\,
%&&\sigma(h_1  h_2 \rightarrow l^{+} l^{-} + X)  =
%\nonumber\\
%&&\sum\limits_{a,b}\,  \int_{x_1}^{1}_1 \int_{x_2}^{1} d_2\, f_{h_1\rightarrow a_1,\as(\mur),\mur,\muf)
%f_{h_2\rightarrow b_2,\as(\mur),\mur,\muf)\nonumber\\}(
%    &&\times  \, \hat{\sigma}^{ab}(\frac{x_1_1},\frac{x_2}{_2};\as(\mur),Q,\muf,\mur)}{
% + { \cal O}\left(\frac{\Lambda^2}{Q^2}\right)\,
%}( d
\label{eq:fact}
%\end{array}
\end{eqnarray}
%}
where the cross section $\sigma$ for 
any hard-scattering inclusive process
%$ab \rightarrow X + all$
is expressed
as a convolution of Parton Distribution Functions (PDFs) $f_a$ and $f_b$
with the partonic cross section
% that describe
%the 
%$\hat{\sigma}^{ab \rightarrow H + X}$.
$\hat{\sigma}^{ab}$.
%
At Leading-Order (LO), the PDFs represent 
the probability of finding a specific parton $a$ ($b$) in the first (second) proton carrying a fraction $x_1$ ($x_2$) of its momentum.
%
Indices $a$ and $b$ in the Eq.~\ref{eq:fact} indicate the various 
kinds of partons,
i.e. gluons, quarks and antiquarks of different flavours, 
that are considered
as the constituents of the proton.
%
The PDFs depend on factorisation scale, $\muf$, while the partonic cross sections depend on the strong coupling,
$\as$, and the factorisation and renormalisation scales,
$\muf$ and $\mur$.
%
The partonic cross sections $\hat\sigma^{ab}$ are calculated in pQCD whereas
PDFs are constrained by global fits to a variety of hard-process experimental data employing
universality of PDFs within a particular factorisation scheme \cite{Collins:2011zzd,Collins:1989gx}.
Recent review articles on PDFs can be found in Refs. \cite{Perez:2012um,Forte:2013wc}. 
%
%PDFs are assumed to be universal such that different scattering reactions can be used 
%to constrain them~\cite{Perez:2012um,Forte:2013wc}.
% in particular one can use specific reaction data 
%for determining the PDFs and then use these PDFs for
%predicting other processes.
%

Accurate determination of PDFs as a function of $x$ requires large amount of
experimental data of a different nature, covering wide kinematic regions
and sensitive to different kinds of partons. The data are provided together with
complex models of correlated uncertainties. The PDFs are determined
from $\chi^2$ fits of the theory predictions to the 
data \cite{MSTWpdf,CT10pdf,NNPDFpdf,Alekhin:2013nda,Jimenez-Delgado:2014twa}. 
Rapid addition of the new data from the LHC experiments and new theory developments 
demand a tool to combined them together in a fast, efficient, open-source platform.
%Measurements of the inclusive Neutral Current (NC) and Charged Current (CC)  
%Deep Inelastic Scattering (DIS) at the $ep$ collider HERA provide crucial information for determining the PDFs.
%The gluon density in small and medium $x$  
%can be accurately determined solely from the HERA data.
%Many processes in $pp$ and $p \bar p$ collisions at LHC and Tevatron, respectively, 
%probe PDFs in the kinematic ranges complementary to the DIS measurements.
%Therefore inclusion of the LHC and Tevatron data in the QCD analysis of the proton structure 
%provide additional constraints on the PDFs, improving either their precision, 
%or providing valuable information on the correlations of PDFs with the fundamental 
%QCD parameters like the strong coupling or the quark masses. 
%In this context, the processes of interest at hadron colliders are
%Drell-Yan (DY) production, $W$-boson asymmetries, associated production of $W$ or $Z$ bosons 
%and heavy quarks, top quark, jet and prompt photon production.
%

This paper describes the open-source QCD fit platform \fitter which includes the set of tools  essential for a comprehensive global 
QCD analysis of $pp$, $p\bar{p}$ and $ep$ scattering processes of the experimental measurement. 
It is developed for determination of PDFs and extraction of fundamental QCD parameters such as the heavy
quark masses and the strong coupling constant. This platform also provides the basis for 
comparisons of different theoretical approaches and can be used for direct tests of the impact 
of new experimental data on the SM parameters in the QCD analyses.

This paper is organised as follows.
%
The structure and overview of \fitter are presented in section~\ref{sec:structure}.
Section~\ref{sec:theory} discusses the various processes available in \fitter
and corresponding theoretical calculations performed in the collinear factorisation using the DGLAP~\cite{Gribov:1972ri,Gribov:1972rt,Lipatov:1974qm,
Dokshitzer:1977sg,Altarelli:1977zs} formalism.
%
Section~\ref{sec:techniques} presents various fast techniques employed by the theory calculations used in \fitter.
Section~\ref{sec:method} elucidates the 
methodology of determining PDFs through fits based on various
% {\bf (what do you mean here
%by approaches?)} 
 $\chi^2$ definitions used in the
minimisation procedure. 
Alternative approaches to the DGLAP formalism are presented in section~\ref{sec:alternative}.
%
\fitter code organisation is discussed in section \ref{sec:organisation}, specific applications 
of the package are given in section~\ref{sec:examples} and the summary is presented in section~\ref{sec:summary}.
%
%{\bf add something more here?.}
