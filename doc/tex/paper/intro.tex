The discovery of the Higgs boson~\cite{Aad:2012tfa,Chatrchyan:2012ufa}
and extensive searches for signals of new physics at the LHC demands accurate precision of the Standard Model (SM) predictions for
hard scattering processes in hadron-hadron collisions.
%\\
%% old proposal:
%In the era of the Higgs discovery~\cite{Aad:2012tfa,Chatrchyan:2012ufa} and extensive searches
%for signals of new physics at the LHC it is crucial
%to have accurate Standard Model (SM) predictions for
%hard scattering processes in hadron-hadron collisions.
The most common approach to calculate the SM cross sections for  
such reactions is to use collinear factorisation in perturbative QCD (pQCD)~\cite{Collins:1989}:
%{\small
\begin{equation}
\small
\begin{array}{lcl}
\sigma(\as,\mur,\muf) & = &
\sum\limits_{a,b}\,  \int\limits_{0}\limits^{1} dx_1\ dx_2\, f_a(x_1,\as,\muf) 
 f_b(x_2,\as,\muf)\\ 
& \times & \, \hat{\sigma}^{ab}(x_1,x_2;\as,\mur,\muf).
\label{eq:fact}
\end{array}
\end{equation}
%}
Here the cross section $\sigma$ for 
%inclusive Higgs production
any hard-scattering inclusive process $ab \rightarrow X + all$
is expressed
as a convolution of Parton Distribution Functions (PDFs) $f_a$ and $f_b$
with the partonic cross section
% that describe
%the 
%$\hat{\sigma}^{ab \rightarrow H + X}$.
$\hat{\sigma}^{ab}$.
%
The PDFs represent 
the probability of finding a specific parton $a$ ($b$) in the first (second) proton carrying a fraction $x_1$ ($x_2$) of its momentum.
%
Indices $a$ and $b$ in the Eq.~\ref{eq:fact} indicates the various 
kinds of partons,
i.e. gluons, quarks and antiquarks of different flavours, 
that are considered
as the constituents of the proton.
%
Both the PDFs and the partonic cross section depend on the strong coupling
$\as$, and the factorisation and renormalisation scales,
$\muf$ and $\mur$, respectively.
%
The partonic cross sections are calculable in pQCD whereas
PDFs cannot be computed analytically in QCD,
they must rather be determined from measurement. 
%
PDFs are assumed to be universal such that different scattering reactions can be used 
to constrain them~\cite{Perez:2012um,Forte:2013wc}.
% in particular one can use specific reaction data 
%for determining the PDFs and then use these PDFs for
%predicting other processes.
%

Measurements of the inclusive Neutral Current (NC) and Charged Current (CC)  
Deep-Inelastic-Scattering (DIS) at the $ep$ collider HERA provide crucial information for determining the PDFs.
%
For instance, the gluon density relevant
for calculating the dominant gluon-gluon fusion contribution to Higgs production
at the LHC can be accurately determined at low and medium $x$ solely from the HERA data.
%
Many processes in $pp$ and $p \bar p$ collisions at LHC and Tevatron, respectively, 
probe PDFs in the kinematic ranges, complementarly to the DIS measurements. 
Therefore inclusion of the LHC and Tevatron data in the QCD analysis of the proton structure 
provide additional constraints on the PDFs, improving either their precision, 
or providing important information of the correlations of PDF with the fundamental 
QCD parameters like strong coupling or quark masses. 
%
%Despite being often plagued by larger perturbative uncertainties,
%
In this context, the processes of interest at hadron colliders are
Drell Yan (DY) production, $W$ asymmetries, associated production of $W$ or $Z$ bosons 
and heavy quarks, top quark, jet and prompt photon production.
%

%\fitter~represents a QCD analysis framework that aims at 
%determining precise PDFs by integrating all the PDF sensitive information
%from HERA, the Tevatron and the LHC.
%
The open-source QCD platform \fitter encloses the set of tools  necessary for a comprehensive global 
QCD analysis of hadron-induced processes even at the early stage of the experimental measurement. 
It has been developed for determination of PDFs and extraction of fundamental QCD parameters such as the heavy
quark masses or the strong coupling constant. This platform also provides the basis for 
comparisons of different theoretical approaches and can be used for direct tests of the impact 
of new experimental data in the QCD analyses.

The outline of this paper is as follows.
%
The structure and overview of \fitter is presented in section~\ref{sec:structure}.
Section~\ref{sec:theory} discusses the various processes 
and corresponding theoretical calculations performed in the DGLAP~\cite{Gribov:1972ri,Gribov:1972rt,Lipatov:1974qm,
Dokshitzer:1977sg,Altarelli:1977zs} formalism that are available in \fitter.
%
Section~\ref{sec:techniques} presents various techniques employed by the theory calculations used in \fitter.
Section~\ref{sec:method} elucidates the 
methodology of determining PDFs through fits based on various
% {\bf (what do you mean here
%by approaches?)} 
 $\chi^2$ definitions used in the
minimisation procedure. 
Alternative approaches to the DGLAP formalism are presented in section~\ref{sec:alternative}.
%
Specific applications of the package are given in
section~\ref{sec:examples} and the summary is presented in section~\ref{sec:summary}.
%
%{\bf add something more here?.}
