%The constant inflow of new experimental measurements 
%with unprecedented accuracy from hadron colliders is a remarkable challenge 
%for the high energy physics community to provide higher-order theory 
%predictions and to develop efficient tools and methods for data analysis.
The recent discovery of the Higgs boson \cite{Aad:2012tfa,Chatrchyan:2012ufa} 
and the extensive searches
for signals of new physics in LHC proton-proton collisions
demand high-precision calculations and computations to test the validity of the Standard Model (SM)
and factorisation in Quantum Chromodynamics (QCD).
Using collinear factorisation, hadron inclusive cross sections may be written as
%
%The discovery of the Higgs boson~\cite{Aad:2012tfa,Chatrchyan:2012ufa}
%and extensive searches for signals of new physics at the LHC demands accurate precision of the Standard Model (SM) predictions for
%hard scattering processes in hadron-hadron collisions.
%\\
%The most common approach to calculate the SM cross sections for  
%such reactions is to use collinear factorisation in perturbative QCD (pQCD)~\cite{Collins:1989}:
%{\small
\begin{eqnarray}
\small
%\begin{array}{lcl}
%&&\sigma(h_1  h_2 \rightarrow l^{+} l^{-} + X)  = 
%\nonumber\\
%&&\sum\limits_{a,b}\,  \int_{x_1}^{1} d\xi_1 \int_{x_2}^{1} d\xi_2\, f_{h_1\rightarrow a}(\xi_1,\as(\mur),\mur,\muf) 
% f_{h_2\rightarrow b}(\xi_2,\as(\mur),\mur,\muf)\nonumber\\
%&&\times  \, \hat{\sigma}^{ab}(\frac{x_1}{\xi_1},\frac{x_2}{\xi_2};\as(\mur),Q,\muf,\mur) 
%+ {\cal O}\left(\frac{\Lambda^2}{Q^2}\right)\,,\\
%
%\sigma(\as(\mur),\mur,\muf)&=& \sum\limits_{a,b}\,  \int\limits_{0}\limits^{1} dx_1\ dx_2 \nonumber \\
%&\times& f_a(x_1,\as(\mur),\muf) 
% f_b(x_2,\as(\mur),\muf) \nonumber \\ 
%&\times&  \, \hat{\sigma}^{ab}(x_1,x_2;\as(\mur),\mur,\muf)\,
\sigma(\as(\mur^2),\mur^2,\muf^2)&=& \sum\limits_{a,b}\,  \int\limits_{0}\limits^{1} dx_1\ dx_2  f_a(x_1,\muf^2) f_b(x_2,\muf^2) \nonumber \\ 
&\times&  \, \hat{\sigma}^{ab}(x_1,x_2;\as(\mur^2),\mur^2,\muf^2)  \nonumber \\
&+&{\cal O}\left(\frac{\Lambda_{QCD}^2}{Q^2}\right)\,
%&&\sigma(h_1  h_2 \rightarrow l^{+} l^{-} + X)  =
%\nonumber\\
%&&\sum\limits_{a,b}\,  \int_{x_1}^{1}_1 \int_{x_2}^{1} d_2\, f_{h_1\rightarrow a_1,\as(\mur),\mur,\muf)
%f_{h_2\rightarrow b_2,\as(\mur),\mur,\muf)\nonumber\\}(
%    &&\times  \, \hat{\sigma}^{ab}(\frac{x_1_1},\frac{x_2}{_2};\as(\mur),Q,\muf,\mur)}{
% + { \cal O}\left(\frac{\Lambda^2}{Q^2}\right)\,
%}( d
\label{eq:fact}
%\end{array}
\end{eqnarray}
%}
where the cross section $\sigma$
% for 
%any hard-scattering inclusive process
%$ab \rightarrow X + all$
is expressed
as a convolution of Parton Distribution Functions (PDFs) $f_a$ and $f_b$
with the parton cross section
% that describe
%the 
%$\hat{\sigma}^{ab \rightarrow H + X}$.
$\hat{\sigma}^{ab}$,  involving a momentum transfer 
$q$ such that $Q^2 = |q^2| \gg \Lambda_{QCD}^2$. 
%
At Leading-Order (LO), the PDFs represent 
the probability of finding a specific parton $a$ ($b$) in the first (second) proton carrying a fraction $x_1$ ($x_2$) of its momentum.
%
The indices $a$ and $b$ in Eq.~\ref{eq:fact} indicate the various 
kinds of partons,
i.e. gluons, quarks and antiquarks of different flavours
that are considered
as the constituents of the proton.
%
The PDFs depend on the factorisation scale, $\muf$, while the parton cross sections depend on the strong coupling,
$\as$, and the factorisation and renormalisation scales,
$\muf$ and $\mur$.
%
The parton cross sections $\hat\sigma^{ab}$ are calculable in perturbative QCD (pQCD) whereas
PDFs are non-perturbative and are usually constrained by global fits to a variety of experimental data. The assumption that PDFs are universal, within a particular factorisation scheme \cite{Collins:1981uw,Collins:1983ju,Collins:1985ue,Collins:1989gx,Collins:2011zzd}, is crucial to this procedure.
Recent review articles on PDFs can be found in Refs. \cite{Perez:2012um,Forte:2013wc}. 
%
%PDFs are assumed to be universal such that different scattering reactions can be used 
%to constrain them~\cite{Perez:2012um,Forte:2013wc}.
% in particular one can use specific reaction data 
%for determining the PDFs and then use these PDFs for
%predicting other processes.
%

A precise determination of PDFs as a function of $x$ requires large amounts of
experimental data that cover a wide kinematic region and that are sensitive to different kinds of partons. Measurements of inclusive Neutral Current (NC) and Charge Current (CC) Deep Inelastic Scattering (DIS) at the lepton-proton ($ep$) collider HERA provide crucial information for determining the PDFs. Different processes in  proton-proton ($pp$) and proton-antiproton ($p \bar p$) collisions at the LHC and the Tevatron, respectively, 
provide complementary information to the DIS measurements.
% The data are provided together with
%complex models of correlated uncertainties.
 The PDFs are determined
from $\chi^2$ fits of the theoretical predictions to the 
data. 
%\cite{MSTWpdf,CT10pdf,NNPDFpdf,Alekhin:2013nda,Jimenez-Delgado:2014twa}. 
The rapid flow of new data from the LHC experiments and the corresponding theoretical developments, which are providing predictions for more complex processes at increasingly higher orders, has motivated the development of a tool to combine them  together in a fast, efficient, open-source platform.
%Measurements of the inclusive Neutral Current (NC) and Charged Current (CC)  
%Deep Inelastic Scattering (DIS) at the $ep$ collider HERA provide crucial information for determining the PDFs.
%The gluon density in small and medium $x$  
%can be accurately determined solely from the HERA data.
%Many processes in $pp$ and $p \bar p$ collisions at LHC and Tevatron, respectively, 
%probe PDFs in the kinematic ranges complementary to the DIS measurements.
%Therefore inclusion of the LHC and Tevatron data in the QCD analysis of the proton structure 
%provide additional constraints on the PDFs, improving either their precision, 
%or providing valuable information on the correlations of PDFs with the fundamental 
%QCD parameters like the strong coupling or the quark masses. 
%In this context, the processes of interest at hadron colliders are
%Drell-Yan (DY) production, $W$-boson asymmetries, associated production of $W$ or $Z$ bosons 
%and heavy quarks, top quark, jet and prompt photon production.
%

This paper describes the open-source QCD fit platform \fitter, which includes a set of tools to facilitate global 
QCD analyses of $pp$, $p\bar{p}$ and $ep$ scattering data. 
It has been developed for the determination of PDFs and the extraction of fundamental parameters of QCD such as the heavy
quark masses and the strong coupling constant. It also provides a common platform for the
comparison of different theoretical approaches. Furthermore, it can be used to test the impact 
of new experimental data on the PDFs and on the SM parameters.

This paper is organised as follows:
%
The general structure of \fitter is presented in section~\ref{sec:structure}.
In section~\ref{sec:theory} the various processes available in \fitter
and the corresponding theoretical calculations, performed within the framework of collinear factorisation and the DGLAP~\cite{Gribov:1972ri,Gribov:1972rt,Lipatov:1974qm,
Dokshitzer:1977sg,Altarelli:1977zs} formalism, are discussed. In
section~\ref{sec:techniques} tools for fast calculations of the theoretical predictions are presented.
In section~\ref{sec:method} the 
methodology to determine PDFs through fits based on various
% {\bf (what do you mean here
%by approaches?)} 
 $\chi^2$ definitions is explained. In particular, different treatments of correlated experimental uncertainties are presented.
Alternative approaches to the DGLAP formalism are presented in section~\ref{sec:alternative}.
%
The organisation of the \fitter code is discussed in section \ref{sec:organisation}, specific applications 
of the package are persented in section~\ref{sec:examples}, which is followed by a summary in section~\ref{sec:summary}.
%
%{\bf add something more here?.}
